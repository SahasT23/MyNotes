\documentclass[a4paper,12pt]{article}

%----------------------------------------------------------------------------------------
%	FONT
%----------------------------------------------------------------------------------------

%----------------------------------------------------------------------------------------
%	PACKAGES
%----------------------------------------------------------------------------------------
\usepackage{url}
\usepackage{parskip} 	
\usepackage{tabularx}

%other packages for formatting
\RequirePackage{color}
\RequirePackage{graphicx}
\usepackage[usenames,dvipsnames]{xcolor}
\usepackage[scale=0.9]{geometry}
\usepackage{amsmath}
\usepackage{tikz}
\usetikzlibrary{shapes,arrows}
\usepackage{rotating}
\usetikzlibrary{shapes.geometric, arrows}
\usepackage{float}
\usepackage{algorithm}
\usepackage{algpseudocode}

\tikzstyle{startstop} = [rectangle, rounded corners, minimum width=3cm, minimum height=1cm,text centered, draw=black, fill=red!30]
\tikzstyle{process} = [rectangle, minimum width=3cm, minimum height=1cm, text centered, draw=black, fill=blue!30]
\tikzstyle{decision} = [diamond, minimum width=3cm, minimum height=1cm, text centered, draw=black, fill=green!30]
\tikzstyle{arrow} = [thick,->,>=stealth]

\definecolor{codegreen}{rgb}{0,0.6,0}
\definecolor{codegray}{rgb}{0.5,0.5,0.5}
\definecolor{codepurple}{rgb}{0.58,0,0.82}
\definecolor{backcolour}{rgb}{0.95,0.95,0.92}

\usepackage{listings}

\lstdefinestyle{mystyle}{
    backgroundcolor=\color{backcolour},   
    commentstyle=\color{codegreen},
    keywordstyle=\color{magenta},
    numberstyle=\tiny\color{codegray},
    stringstyle=\color{codepurple},
    basicstyle=\ttfamily\footnotesize,
    breakatwhitespace=false,         
    breaklines=true,                 
    captionpos=b,                    
    keepspaces=true,                 
    numbers=left,                    
    numbersep=5pt,                  
    showspaces=false,                
    showstringspaces=false,
    showtabs=false,                  
    tabsize=2
}

\lstset{style=mystyle}


%tabularx environment
\usepackage{tabularx}

%for lists within experience section
\usepackage{enumitem}

% centered version of 'X' col. type
\newcolumntype{C}{>{\centering\arraybackslash}X} 

%to prevent spillover of tabular into next pages
\usepackage{supertabular}
\usepackage{tabularx}
\newlength{\fullcollw}
\setlength{\fullcollw}{0.47\textwidth}

%custom \section
\usepackage{titlesec}				
\usepackage{multicol}
\usepackage{multirow}

%CV Sections inspired by: 
%http://stefano.italians.nl/archives/26
\titleformat{\section}{\large\scshape\raggedright}{}{0em}{}[\titlerule]
\titlespacing{\section}{0pt}{10pt}{10pt}

%for publications
\usepackage[style=authoryear,sorting=ynt, maxbibnames=2]{biblatex}

%Setup hyperref package, and colours for links
\usepackage[unicode, draft=false]{hyperref}
\definecolor{linkcolour}{rgb}{0,0.2,0.6}
\hypersetup{colorlinks,breaklinks,urlcolor=linkcolour,linkcolor=linkcolour}
\addbibresource{citations.bib}
\setlength\bibitemsep{1em}

%for social icons
\usepackage{fontawesome5}

%debug page outer frames
%\usepackage{showframe}
%----------------------------------------------------------------------------------------
%	BEGIN DOCUMENT
%----------------------------------------------------------------------------------------
\begin{document}

% non-numbered pages
\pagestyle{empty}

\section{C Multi Choice Answers and Explanations}

\begin{enumerate}
    \item Answer C
    \item Answer C
    \item Answer: A\\
    Explanation: The assignment \verb|x = 0| always evaluates to true (non-zero), making the condition true.
    \item Answer: B, trying to access \textbf{index} 5 instead of \textbf{element} 5.
    \item Answer: C
    \item Answer: A
    \item Answer: A\\
    Explanation: Array indices in C start at 0, so \verb|arr[i]| is out-of-bounds when \verb|i = 5|.
    \item Answer: B
    \item Answer: A
    \item Answer: A\\
    Explanation: The file is opened in read mode, but writing is attempted.
\end{enumerate}

\section{C++ Multi Choice Answers and Explanations}

\begin{enumerate}
    \item Answer: D
    \item Answer: A
    \item Answer: A
    \item Answer: A
    \item Answer: B
    \item Answer: A\\
    Explanation: The correct syntax is \verb|delete[] ptr;| for arrays.
    \item Answer: B
    \item Answer: D
    \item Answer: B \verb|delete[] ptr;|
    \item Answer: B
\end{enumerate}

\newpage

\section{C Creative Solution}

\lstset{language=C}
\begin{lstlisting}
#include <stdio.h>

// Function to perform linear search
int linearSearch(int arr[], int size, int value) {
    for (int i = 0; i < size; i++) {
        if (arr[i] == value) {
            return i; // Return index if found
        }
    }
    return -1; // Return -1 if not found
}

// Function to find maximum and minimum in the array
void findMaxAndMin(int arr[], int size, int *max, int *min) {
    *max = arr[0];
    *min = arr[0];
    for (int i = 1; i < size; i++) {
        if (arr[i] > *max) {
            *max = arr[i];
        }
        if (arr[i] < *min) {
            *min = arr[i];
        }
    }
}

// Function to calculate sum and average
void calculateSumAndAverage(int arr[], int size, int *sum, float *average) {
    *sum = 0;
    for (int i = 0; i < size; i++) {
        *sum += arr[i];
    }
    *average = (float)(*sum) / size;
}

int main() {
    // Define and initialize the array
    int arr[10] = {12, 45, 78, 34, 23, 56, 89, 90, 67, 33};
    int size = 10;
    int searchValue, index, max, min, sum;
    float average;

    // Print the array
    printf("Array elements: ");
    for (int i = 0; i < size; i++) {
        printf("%d ", arr[i]);
    }
    printf("\n");

    // Perform linear search
    printf("Enter a value to search: ");
    scanf("%d", &searchValue);
    index = linearSearch(arr, size, searchValue);
    if (index != -1) {
        printf("Value found at index %d.\n", index);
    } else {
        printf("Value not found in the array.\n");
    }

    // Continued on the next page

    // Find and print the maximum and minimum values
    findMaxAndMin(arr, size, &max, &min);
    printf("Maximum value: %d\n", max);
    printf("Minimum value: %d\n", min);

    // Calculate and print the sum and average
    calculateSumAndAverage(arr, size, &sum, &average);
    printf("Sum of array elements: %d\n", sum);
    printf("Average of array elements: %.2f\n", average);

    return 0;
}


\end{lstlisting}

\begin{table}[h!]
\centering
\begin{tabular}{|l|c|p{8cm}|}
\hline
\textbf{Task} & \textbf{Marks} & \textbf{Explanation} \\ \hline
Define and initialize the array & 4 & The array must contain the specified values and be printed correctly. \\ \hline
Implement \texttt{linearSearch} function & 8 & Function must iterate through the array and return the correct index or \texttt{-1}. \\ \hline
Use \texttt{linearSearch} in \texttt{main} and print results & 8 & User input must be handled, and results displayed correctly. \\ \hline
Implement \texttt{findMaxAndMin} function & 6 & Function must calculate and return the correct maximum and minimum values. \\ \hline
Implement sum and average calculation & 4 & Function must return and print the sum and average correctly. \\ \hline
\end{tabular}
\caption{Marking scheme for the C program question}
\label{tab:marking_scheme_c}
\end{table}


\newpage

\section{C++ Creative Solution}

\begin{lstlisting}
#include <iostream>
#include <string>
using namespace std;

// Student class definition
class Student {
private:
    string name;
    float grades[5];
    int rollNumber;

public:
    // Constructor to initialize attributes
    Student(string studentName, float studentGrades[], int studentRollNumber) {
        name = studentName;
        rollNumber = studentRollNumber;
        for (int i = 0; i < 5; i++) {
            grades[i] = studentGrades[i];
        }
    }

    // Function to calculate average grade
    float calculateAverage() {
        float sum = 0.0;
        for (int i = 0; i < 5; i++) {
            sum += grades[i];
        }
        return sum / 5;
    }

    // Function to check pass/fail
    bool isPass() {
        return calculateAverage() >= 50;
    }

    // Function to display student details
    void displayDetails() {
        cout << "Name: " << name << endl;
        cout << "Roll Number: " << rollNumber << endl;
        cout << "Grades: ";
        for (int i = 0; i < 5; i++) {
            cout << grades[i] << " ";
        }
        cout << endl;
        cout << "Average: " << calculateAverage() << endl;
        cout << (isPass() ? "Status: Pass" : "Status: Fail") << endl;
    }
};

int main() {
    float grades1[5] = {70, 80, 90, 60, 50};
    Student student1("John Doe", grades1, 101);

    // Display details and status
    student1.displayDetails();

    return 0;
}

\end{lstlisting}

\begin{table}[H]
\centering
\begin{tabular}{|l|c|p{8cm}|}
\hline
\textbf{Task} & \textbf{Marks} & \textbf{Explanation} \\ \hline
Define the class with required members & 5 & Class should include \texttt{name}, \texttt{grades}, \texttt{rollNumber}, and appropriate access. \\ \hline
Implement \texttt{calculateAverage} function & 5 & Function must calculate the average of the grades correctly. \\ \hline
Implement \texttt{isPass} function & 5 & Logic for checking if average is $\geq 50$ should be correct. \\ \hline
Implement \texttt{displayDetails} function & 5 & Should display all required fields in a formatted manner. \\ \hline
Correctly instantiate objects in \texttt{main} & 5 & Objects must be created with provided details. \\ \hline
Call functions and test program & 5 & Program output should match the requirements. \\ \hline
\end{tabular}
\caption{Marking scheme for the C++ class question}
\label{tab:marking_scheme_cpp}
\end{table}


\vfill
\end{document}