\documentclass[a4paper,12pt]{article}

%----------------------------------------------------------------------------------------
%	FONT
%----------------------------------------------------------------------------------------

%----------------------------------------------------------------------------------------
%	PACKAGES
%----------------------------------------------------------------------------------------
\usepackage{url}
\usepackage{parskip} 	
\usepackage{tabularx}

%other packages for formatting
\RequirePackage{color}
\RequirePackage{graphicx}
\usepackage[usenames,dvipsnames]{xcolor}
\usepackage[scale=0.9]{geometry}
\usepackage{amsmath}
\usepackage{tikz}
\usetikzlibrary{shapes,arrows}
\usepackage{rotating}
\usetikzlibrary{shapes.geometric, arrows}
\usepackage{float}
\usepackage{algorithm}
\usepackage{algpseudocode}

\tikzstyle{startstop} = [rectangle, rounded corners, minimum width=3cm, minimum height=1cm,text centered, draw=black, fill=red!30]
\tikzstyle{process} = [rectangle, minimum width=3cm, minimum height=1cm, text centered, draw=black, fill=blue!30]
\tikzstyle{decision} = [diamond, minimum width=3cm, minimum height=1cm, text centered, draw=black, fill=green!30]
\tikzstyle{arrow} = [thick,->,>=stealth]

\definecolor{codegreen}{rgb}{0,0.6,0}
\definecolor{codegray}{rgb}{0.5,0.5,0.5}
\definecolor{codepurple}{rgb}{0.58,0,0.82}
\definecolor{backcolour}{rgb}{0.95,0.95,0.92}

\usepackage{listings}
\lstdefinestyle{mystyle}{
    backgroundcolor=\color{backcolour},   
    commentstyle=\color{codegreen},
    keywordstyle=\color{magenta},
    numberstyle=\tiny\color{codegray},
    stringstyle=\color{codepurple},
    basicstyle=\ttfamily\footnotesize,
    breakatwhitespace=false,         
    breaklines=true,                 
    captionpos=b,                    
    keepspaces=true,                 
    numbers=left,                    
    numbersep=5pt,                  
    showspaces=false,                
    showstringspaces=false,
    showtabs=false,                  
    tabsize=2
}

\lstset{style=mystyle}

%tabularx environment
\usepackage{tabularx}

%for lists within experience section
\usepackage{enumitem}

% centered version of 'X' col. type
\newcolumntype{C}{>{\centering\arraybackslash}X} 

%to prevent spillover of tabular into next pages
\usepackage{supertabular}
\usepackage{tabularx}
\newlength{\fullcollw}
\setlength{\fullcollw}{0.47\textwidth}

%custom \section
\usepackage{titlesec}				
\usepackage{multicol}
\usepackage{multirow}

%CV Sections inspired by: 
%http://stefano.italians.nl/archives/26
\titleformat{\section}{\large\scshape\raggedright}{}{0em}{}[\titlerule]
\titlespacing{\section}{0pt}{10pt}{10pt}

%for publications
\usepackage[style=authoryear,sorting=ynt, maxbibnames=2]{biblatex}

%Setup hyperref package, and colours for links
\usepackage[unicode, draft=false]{hyperref}
\definecolor{linkcolour}{rgb}{0,0.2,0.6}
\hypersetup{colorlinks,breaklinks,urlcolor=linkcolour,linkcolor=linkcolour}
\addbibresource{citations.bib}
\setlength\bibitemsep{1em}

%for social icons
\usepackage{fontawesome5}

%debug page outer frames
%\usepackage{showframe}
%----------------------------------------------------------------------------------------
%	BEGIN DOCUMENT
%----------------------------------------------------------------------------------------
\begin{document}

% non-numbered pages
\pagestyle{empty} 

%----------------------------------------------------------------------------------------
%	TITLE
%----------------------------------------------------------------------------------------
% Title Page
\begin{titlepage}
    \centering
    \vspace*{2cm}
    \Huge{\textbf{C++ Programming For Absolute Beginners \& Question Set II}} \\[1.5cm]
    \Large{Sahas Talasila} \\[1cm]
    \vfill
    \vfill
\end{titlepage}

% Table of Contents
\tableofcontents
\newpage

\section{Pointers in C++}

In C++, a \textbf{pointer} is a variable that holds the memory address of another variable. Pointers are powerful because they allow you to:
\begin{itemize}
    \item Directly access and modify memory locations.
    \item Pass large data structures to functions efficiently by reference (address) rather than by value (copy).
    \item Dynamically allocate and deallocate memory on the heap as needed, giving more control over how memory is used.
\end{itemize}

However, with great power comes great responsibility. Incorrect use of pointers can lead to errors such as:
\begin{itemize}
    \item \textbf{Memory leaks:} Forgetting to release dynamically allocated memory.
    \item \textbf{Dangling pointers:} Pointers that still reference memory that has already been deallocated.
    \item \textbf{Segmentation faults (crashes):} Accessing invalid memory.
\end{itemize}

\subsection{Pointers and Memory: A Visual Overview}
To better understand pointers, let's visualize how variables and pointers map to memory. Suppose we have a simple scenario:

\begin{lstlisting}[language=C++]
int x = 10;    // x is stored in some memory location
int* ptr = &x; // ptr stores the address of x
\end{lstlisting}

Conceptually, in memory, it might look like this:

\begin{verbatim}
Memory Address      Value
     0x100         10    <--- This is x
     0x104         ... 
     0x108         ...
     0x10C   -->  0x100  <--- This is ptr (contains address 0x100)
\end{verbatim}

Here, \texttt{x} resides at address \texttt{0x100} (example only), and \texttt{ptr} at address \texttt{0x10C}. The contents of \texttt{ptr} are \texttt{0x100}, meaning ``pointer \texttt{ptr} holds the address of \texttt{x}.''

\subsection{Declaration and Initialization}

A pointer is declared by specifying the type of data it will point to, followed by an asterisk (\texttt{*}) and a pointer name. For example:

\begin{lstlisting}[language=C++]
int* ptr;     // Pointer to an integer
double* dptr; // Pointer to a double
\end{lstlisting}

You can also write \texttt{int *ptr;} or \texttt{int * ptr;}, which are stylistically the same in C++. To obtain the address of a variable, use the \textbf{\&} (address-of) operator:

\begin{lstlisting}[language=C++]
int x = 10;
int* ptr = &x; // ptr now holds the address of x
\end{lstlisting}

\paragraph{Common Error: Uninitialized Pointer}
\begin{itemize}
    \item \textbf{What happens?} If you declare a pointer but never assign it an address, it may point to a random memory location.
    \item \textbf{Error symptom:} Attempting to dereference an uninitialized pointer often causes a \textit{segmentation fault} or undefined behavior.
    \item \textbf{How to fix:} Always initialize pointers, for example:
    \begin{lstlisting}
    int* ptr = nullptr;  // or int* ptr = &someVariable;
    \end{lstlisting}
\end{itemize}

\subsection{Dereferencing a Pointer}

To access the \textit{value} stored at the address contained in a pointer, use the \textbf{*} (dereference) operator:

\begin{lstlisting}[language=C++]
int x = 10;
int* ptr = &x;
std::cout << *ptr << std::endl; // Prints the value of x (10)
\end{lstlisting}

Changing the value through a pointer:
\begin{lstlisting}[language=C++]
*ptr = 20;  // Now x = 20
\end{lstlisting}

Visually:

\begin{verbatim}
   ptr ------> x (value = 10)

   *ptr is the same as x, so *ptr = 20 modifies x to 20.
\end{verbatim}

\paragraph{Common Error: Null Pointer Dereference}
\begin{itemize}
    \item \textbf{What happens?} If \texttt{ptr} is \texttt{nullptr} (or \texttt{NULL}), dereferencing it (\texttt{*ptr}) leads to invalid memory access.
    \item \textbf{Error symptom:} Typically results in a crash (segmentation fault).
    \item \textbf{How to fix:} Always check if a pointer is \texttt{nullptr} before dereferencing:
    \begin{lstlisting}[language=C++]
    if (ptr != nullptr) {
        *ptr = 100;
    }
    \end{lstlisting}
\end{itemize}

\subsection{Dynamic Memory Allocation}

\textbf{Dynamic memory allocation} allows you to request memory from the \emph{heap} at runtime. This is useful if the size of an array or the lifespan of an object is not known until the program is running.  

In C++, dynamic memory is allocated using the \textbf{\texttt{new}} operator and deallocated using the \textbf{\texttt{delete}} operator.

\subsubsection{Allocating and Deallocating a Single Object}

\begin{lstlisting}[language=C++]
int* ptr = new int; // Allocates an integer on the heap
*ptr = 42;          // Store a value
std::cout << *ptr << std::endl; // Output: 42

delete ptr;   // Deallocate memory to prevent memory leak
ptr = nullptr; // Optional, but good practice to avoid dangling pointer
\end{lstlisting}

Diagrammatically:

\begin{verbatim}
   Stack                      Heap
   -----                      ----
   ptr  --------------------> [  42  ]
                             (allocated by new)
   ... other local variables
\end{verbatim}

When you do \texttt{delete ptr;}:

\begin{verbatim}
  * Freed memory on the heap *
  ptr still has the old address,
  but that memory is no longer valid
\end{verbatim}

Setting \texttt{ptr = nullptr;} means \texttt{ptr} is no longer pointing to invalid memory.

\subsubsection{Allocating and Deallocating Arrays}

\begin{lstlisting}[language=C++]
int* arr = new int[5]; // Allocates an array of 5 ints on the heap
for(int i = 0; i < 5; i++) {
    arr[i] = i * 2; // Assign values
}
delete[] arr; // Use delete[] for arrays
arr = nullptr;
\end{lstlisting}

Visually:

\begin{verbatim}
   Stack                  Heap
   -----                  ----
   arr ---------------> [ 0  | 2  | 4  | 6  | 8  ]
                         size=5   (allocated by new[])
\end{verbatim}

\paragraph{Common Error: Using \texttt{delete} instead of \texttt{delete[]}}
\begin{itemize}
    \item \textbf{What happens?} If you allocated with \texttt{new[]} but used \texttt{delete} (without brackets) to deallocate, it leads to undefined behavior.
    \item \textbf{Error symptom:} May cause crashes or corrupt your memory.
    \item \textbf{How to fix:} Always match \texttt{new[]} with \texttt{delete[]}:
    \begin{lstlisting}
    int* arr = new int[5];
    delete[] arr;  // correct
    \end{lstlisting}
\end{itemize}

\subsection{Memory Leaks and Dangling Pointers}

\begin{itemize}
    \item A \textbf{memory leak} occurs when you lose the address of allocated memory without deallocating it.  
    \item A \textbf{dangling pointer} occurs when a pointer points to memory that has already been freed.
\end{itemize}

\paragraph{Example: Memory Leak}

\begin{lstlisting}[language=C++]
int* leakPtr = new int(100);
// Some code that never calls delete on leakPtr
// Eventually, leakPtr goes out of scope or is reassigned
// The memory allocated for '100' is never freed => Memory leak
\end{lstlisting}

\paragraph{Example: Dangling Pointer}

\begin{lstlisting}[language=C++]
int* dangPtr = new int(50);
delete dangPtr;
// dangPtr still points to that memory address, but it's invalid now.
std::cout << *dangPtr << std::endl; // undefined behavior
\end{lstlisting}

\noindent
\textbf{How to Avoid:}
\begin{itemize}
    \item Always pair \texttt{new} with \texttt{delete} and \texttt{new[]} with \texttt{delete[]}.
    \item Set pointers to \texttt{nullptr} after deleting them.
    \item Use RAII (Resource Acquisition Is Initialization) techniques, smart pointers (\texttt{std::unique\_ptr}, \texttt{std::shared\_ptr}), or other memory management strategies in modern C++.
\end{itemize}

\subsection{Smart Pointers (Brief Mention)}
In modern C++, you can avoid many pointer pitfalls by using \textit{smart pointers} from the \texttt{<memory>} header, such as:
\begin{itemize}
    \item \textbf{\texttt{std::unique\_ptr<T>}}: Exclusively owns the resource, automatically deletes it when out of scope.
    \item \textbf{\texttt{std::shared\_ptr<T>}}: Allows multiple owners of a resource, deleted when the last reference goes out of scope.
\end{itemize}

These can prevent memory leaks and dangling pointers because the deletion is handled automatically by RAII.

\subsection{Summary of Pointer Errors and Resolutions}

\begin{table}[h!]
\centering
\caption{Common Pointer-Related Errors}
\begin{tabular}{|l|p{5cm}|p{5cm}|}
\hline
\textbf{Error} & \textbf{Cause} & \textbf{Solution} \\ \hline
Uninitialized Pointer & Declaring a pointer but never assigning an address & Initialize pointer to \texttt{nullptr} or a valid address \\ \hline
Null Pointer Dereference & Dereferencing a pointer that is \texttt{nullptr} & Check if pointer is \texttt{nullptr} before dereferencing \\ \hline
Memory Leak & Not calling \texttt{delete}/\texttt{delete[]} for dynamically allocated memory & Ensure each \texttt{new}/\texttt{new[]} has a matching \texttt{delete}/\texttt{delete[]} \\ \hline
Dangling Pointer & Pointer still references memory that is already freed & Set pointer to \texttt{nullptr} after deletion, avoid using it afterwards \\ \hline
Mismatched Delete & Using \texttt{delete} on memory allocated with \texttt{new[]} & Always use \texttt{delete} for single objects, \texttt{delete[]} for arrays \\ \hline
\end{tabular}
\end{table}

\newpage
\subsection{Example: Safe Usage of Dynamic Memory}
Below is an example program illustrating safe pointer usage, dynamic allocation, and deallocation:

\begin{lstlisting}[language=C++]
#include <iostream>

int main() {
    // 1. Allocate a single integer
    int* singleInt = new int;
    *singleInt = 42;
    std::cout << "singleInt value: " << *singleInt << std::endl;

    // 2. Allocate an array
    int size;
    std::cout << "Enter the number of elements: ";
    std::cin >> size;

    int* arr = new int[size];
    for(int i = 0; i < size; i++) {
        arr[i] = i * 10;
    }

    std::cout << "Array elements: ";
    for(int i = 0; i < size; i++) {
        std::cout << arr[i] << " ";
    }
    std::cout << std::endl;

    // 3. Deallocate both
    delete singleInt;
    singleInt = nullptr;

    delete[] arr;
    arr = nullptr;

    return 0;
}
\end{lstlisting}

\noindent
\textbf{Key Takeaways}:
\begin{itemize}
    \item We used \texttt{new} and \texttt{delete} correctly.
    \item We used \texttt{new[]} and \texttt{delete[]} correctly.
    \item Setting pointers to \texttt{nullptr} after deletion helps avoid accidental usage later.
\end{itemize}


\section{Multiple Choice Questions (MCQs)}

\begin{enumerate}
    \item Which operator is used to allocate memory dynamically in C++?
    \begin{enumerate}
        \item \texttt{malloc()}
        \item \textbf{\texttt{new}}
        \item \texttt{alloc()}
        \item \texttt{malloc\_cpp()}
    \end{enumerate}

    \item Which of the following is the correct way to declare a pointer to an integer?
    \begin{enumerate}
        \item \texttt{int ptr;}
        \item \textbf{\texttt{int* ptr;}}
        \item \texttt{int ptr*;}
        \item \texttt{pointer<int> ptr;}
    \end{enumerate}

    \item Which symbol is used to get the address of a variable?
    \begin{enumerate}
        \item \texttt{*}
        \item \textbf{\texttt{\&}}
        \item \texttt{\%}
        \item \texttt{\#}
    \end{enumerate}

    \item When using \texttt{delete} on an array allocated by \texttt{new}, which syntax should be used?
    \begin{enumerate}
        \item \texttt{delete arr;}
        \item \textbf{\texttt{delete[] arr;}}
        \item \texttt{free(arr);}
        \item \texttt{arr.delete();}
    \end{enumerate}

    \item What is the result of dereferencing a pointer that has not been initialized?
    \begin{enumerate}
        \item It points to the first memory location in the program.
        \item It always gives 0.
        \item \textbf{Undefined behavior.}
        \item It produces a compile-time error.
    \end{enumerate}

    \item Which of the following statements about dynamic memory allocation is true?
    \begin{enumerate}
        \item Memory is allocated on the stack.
        \item You must free it explicitly in Java.
        \item \textbf{Memory is allocated on the heap and must be freed explicitly in C++.}
        \item It is managed automatically by the C++ runtime.
    \end{enumerate}

    \item What happens if you use \texttt{delete} on a pointer that was not allocated by \texttt{new}?
    \begin{enumerate}
        \item The program ignores the request.
        \item \textbf{The program exhibits undefined behavior.}
        \item The pointer is automatically set to \texttt{nullptr}.
        \item Nothing happens.
    \end{enumerate}

    \item Which of the following is \textbf{not} a benefit of using pointers?
    \begin{enumerate}
        \item They can allow dynamic memory management.
        \item They can be used to pass large objects to functions efficiently.
        \item \textbf{They automatically prevent memory leaks.}
        \item They can help with implementing data structures like linked lists.
    \end{enumerate}

    \item If \texttt{p} is a pointer to an integer, what does the expression \texttt{p + 3} mean?
    \begin{enumerate}
        \item Add 3 bytes to the address in \texttt{p}.
        \item \textbf{Point to the 3rd integer ahead of \texttt{p}. (According to the size of \texttt{int}.)}
        \item Add 3 to the value pointed to by \texttt{p}.
        \item This is invalid syntax.
    \end{enumerate}

    \item Which of the following best describes a \textbf{dangling pointer}?
    \begin{enumerate}
        \item A pointer that has a \texttt{nullptr} value.
        \item A pointer that was declared but never initialized.
        \item \textbf{A pointer that points to memory that has been freed or deallocated.}
        \item A pointer that points to an array's first element.
    \end{enumerate}

\end{enumerate}

\section{Short Form Questions}

\begin{enumerate}
    \item What does the \texttt{*} operator do when placed in front of a pointer variable?
    \item How do you obtain the address of a variable in C++?
    \item Which C++ operator is used to free dynamically allocated memory for a single object?
    \item Which C++ operator is used to allocate memory for an array?
    \item What is a \textbf{memory leak}?
    \item Define a \textbf{dangling pointer}.
    \item Why should we set a pointer to \texttt{nullptr} after deleting it?
    \item What is the difference between \texttt{new} and \texttt{new[]}?
    \item How do you dynamically allocate a single \texttt{double} variable in C++?
    \item Which keyword can you use instead of \texttt{NULL} in modern C++ to indicate an empty pointer?
\end{enumerate}

\section{Open Ended (Coding) Questions}

\begin{enumerate}
    \item \textbf{Dynamic Array Function:} 
    Write a C++ function \texttt{createArray} that:
    \begin{itemize}
        \item Dynamically allocates an integer array of size \texttt{n}.
        \item Fills the array with values from \texttt{1} to \texttt{n}.
        \item Returns the pointer to the array.
    \end{itemize}
    Then, write a \texttt{main} function that calls \texttt{createArray}, prints the values, and frees the memory.

    \item \textbf{Singly Linked List Implementation:}
    Implement a singly linked list class in C++ with the following operations:
    \begin{itemize}
        \item \texttt{pushFront(int val)}: Inserts a new node at the beginning.
        \item \texttt{popFront()}: Removes the first node (if it exists).
        \item \texttt{printList()}: Prints all node values.
        \item Proper destructor to free all nodes.
    \end{itemize}
    Demonstrate these operations in \texttt{main}.

    \item \textbf{2D Dynamic Array (Double Pointer):}
    Write a C++ program that:
    \begin{itemize}
        \item Dynamically allocates a 2D integer array of size \texttt{rows} x \texttt{cols}.
        \item Fills it with some sample values (e.g., row index + column index).
        \item Prints the entire 2D array.
        \item Frees the allocated memory.
    \end{itemize}

    \item \textbf{Reading Unknown Number of Integers:}
    Create a program that:
    \begin{itemize}
        \item Continuously reads integer input from \texttt{std::cin} until a non-integer or end-of-file is encountered.
        \item Stores these integers in a dynamically allocated array that grows as needed (e.g., double the size whenever full).
        \item Prints all the integers read and then deallocates the memory.
    \end{itemize}

    \item \textbf{Smart Pointer Demonstration:}
    Write a short code snippet that (\textbf{EXTRA}):
    \begin{itemize}
        \item Creates a \texttt{std::unique\_ptr<int>} and allocates an integer.
        \item Assigns a value and prints it.
        \item Shows that when the \texttt{unique\_ptr} goes out of scope, memory is automatically freed.
    \end{itemize}
    Compare this usage to a raw pointer to highlight how smart pointers help prevent memory leaks.
\end{enumerate}

\section{Spot the Error}

\begin{enumerate}
    \item Spot the error in this pointer initialization:
\begin{lstlisting}
int *ptr;
*ptr = 10;
\end{lstlisting}

    \item Spot the error in this dynamic memory allocation:
\begin{lstlisting}
int *arr = malloc(5 * sizeof(int));
arr[0] = 10;
\end{lstlisting}

    \item Spot the error in this deallocation logic:
\begin{lstlisting}
int *p = new int;
delete p[];
\end{lstlisting}

    \item Spot the error in using a pointer after deletion:
\begin{lstlisting}
int *data = new int(42);
delete data;
std::cout << *data;
\end{lstlisting}

    \item Spot the error in freeing stack memory:
\begin{lstlisting}
int x = 100;
free(&x);
\end{lstlisting}

    \item Spot the error in this memory leak scenario:
\begin{lstlisting}
int *ptr = new int(10);
ptr = new int(20);
\end{lstlisting}

    \item Spot the error in this null pointer dereference:
\begin{lstlisting}
int *p = nullptr;
*p = 5;
\end{lstlisting}

    \item Spot the error in this reallocation example:
\begin{lstlisting}
int *arr = (int*) malloc(3 * sizeof(int));
arr = realloc(arr, 6 * sizeof(int));
if (arr == NULL)
    std::cout << "Reallocation failed";
\end{lstlisting}
\end{enumerate}

\end{document}