\documentclass[a4paper,12pt]{article}

%----------------------------------------------------------------------------------------
%	FONT
%----------------------------------------------------------------------------------------

%----------------------------------------------------------------------------------------
%	PACKAGES
%----------------------------------------------------------------------------------------
\usepackage{url}
\usepackage{parskip} 	
\usepackage{tabularx}

%other packages for formatting
\RequirePackage{color}
\RequirePackage{graphicx}
\usepackage[usenames,dvipsnames]{xcolor}
\usepackage[scale=0.9]{geometry}
\usepackage{amsmath}
\usepackage{tikz}
\usetikzlibrary{shapes,arrows}
\usepackage{rotating}
\usetikzlibrary{shapes.geometric, arrows}
\usepackage{float}
\usepackage{algorithm}
\usepackage{algpseudocode}

\tikzstyle{startstop} = [rectangle, rounded corners, minimum width=3cm, minimum height=1cm,text centered, draw=black, fill=red!30]
\tikzstyle{process} = [rectangle, minimum width=3cm, minimum height=1cm, text centered, draw=black, fill=blue!30]
\tikzstyle{decision} = [diamond, minimum width=3cm, minimum height=1cm, text centered, draw=black, fill=green!30]
\tikzstyle{arrow} = [thick,->,>=stealth]

\definecolor{codegreen}{rgb}{0,0.6,0}
\definecolor{codegray}{rgb}{0.5,0.5,0.5}
\definecolor{codepurple}{rgb}{0.58,0,0.82}
\definecolor{backcolour}{rgb}{0.95,0.95,0.92}

\usepackage{listings}

\lstdefinestyle{mystyle}{
    backgroundcolor=\color{backcolour},   
    commentstyle=\color{codegreen},
    keywordstyle=\color{magenta},
    numberstyle=\tiny\color{codegray},
    stringstyle=\color{codepurple},
    basicstyle=\ttfamily\footnotesize,
    breakatwhitespace=false,         
    breaklines=true,                 
    captionpos=b,                    
    keepspaces=true,                 
    numbers=left,                    
    numbersep=5pt,                  
    showspaces=false,                
    showstringspaces=false,
    showtabs=false,                  
    tabsize=2
}

\lstset{style=mystyle}


%tabularx environment
\usepackage{tabularx}

%for lists within experience section
\usepackage{enumitem}

% centered version of 'X' col. type
\newcolumntype{C}{>{\centering\arraybackslash}X} 

%to prevent spillover of tabular into next pages
\usepackage{supertabular}
\usepackage{tabularx}
\newlength{\fullcollw}
\setlength{\fullcollw}{0.47\textwidth}

%custom \section
\usepackage{titlesec}				
\usepackage{multicol}
\usepackage{multirow}

%CV Sections inspired by: 
%http://stefano.italians.nl/archives/26
\titleformat{\section}{\large\scshape\raggedright}{}{0em}{}[\titlerule]
\titlespacing{\section}{0pt}{10pt}{10pt}

%for publications
\usepackage[style=authoryear,sorting=ynt, maxbibnames=2]{biblatex}

%Setup hyperref package, and colours for links
\usepackage[unicode, draft=false]{hyperref}
\definecolor{linkcolour}{rgb}{0,0.2,0.6}
\hypersetup{colorlinks,breaklinks,urlcolor=linkcolour,linkcolor=linkcolour}
\addbibresource{citations.bib}
\setlength\bibitemsep{1em}

%for social icons
\usepackage{fontawesome5}

%debug page outer frames
%\usepackage{showframe}
%----------------------------------------------------------------------------------------
%	BEGIN DOCUMENT
%----------------------------------------------------------------------------------------
\begin{document}

\title{C and C++ Programming Assessment 1}
\maketitle

\newpage
% non-numbered pages
\pagestyle{empty}

\section{C Multiple Choice}

Each question will have four options, with each multiple choice question being worth \textbf{Two marks}.

\begin{enumerate}
    \item What is the size of an integer on most systems?
    \begin{itemize}
        \item A) 1 byte
        \item B) 2 bytes
        \item C) 4 bytes
        \item D) 8 bytes
    \end{itemize}

    \item Which of the following is NOT a valid keyword in C?
    \begin{itemize}
        \item A) auto
        \item B) struct
        \item C) class
        \item D) enum
    \end{itemize}

    \item What will this code output?

    \lstset{language=C}
    \begin{lstlisting}
        int x = 10;
        
        if (x = 0) {
            printf("Zero");
            } else {
    
        printf("Non-Zero");
    }
    
    \end{lstlisting}    
    
    \begin{itemize}
        \item A) Zero
        \item B) Non-Zero
        \item C) Compilation Error
        \item D) Undefined Behavior
    \end{itemize}

    \item What is wrong with the following code?

    \lstset{language=C}
    \begin{lstlisting}
        int arr[5] = {1, 2, 3, 4, 5};
        printf("%d", arr[5]);
    \end{lstlisting}    
    
    \begin{itemize}
        \item A) Array size is not specified
        \item B) Out-of-bounds access of the array
        \item C) Incorrect format specifier
        \item D) Nothing is wrong
    \end{itemize}

    \item Which of the following loops executes at least once?
    \begin{itemize}
        \item A) for
        \item B) while
        \item C) do-while
        \item D) None of the above
    \end{itemize}

    \item Which of these operations cannot be performed directly with arrays in C?
    \begin{itemize}
        \item A) Assignment
        \item B) Accessing elements
        \item C) Iteration
        \item D) Indexing
    \end{itemize}

    \item What is wrong with the following logic for summing array elements?

    \lstset{language=C}
    \begin{lstlisting}
        int arr[5] = {1, 2, 3, 4, 5};
        int sum = 0;
        
        for (int i = 1; i <= 5; i++) {
            sum += arr[i];
        }
    \end{lstlisting}

    \begin{itemize}
        \item A) Loop bounds are incorrect
        \item B) Missing initialization of sum
        \item C) Array is not declared properly
        \item D) Nothing is wrong
    \end{itemize}

    \item What is the purpose of the \verb|auto| keyword in C?
    \begin{itemize}
        \item A) Allocate memory dynamically
        \item B) Specify a variable's scope as local
        \item C) Make a variable global
        \item D) None of the above
    \end{itemize}

    \item Which of the following best describes pointers in C?
    \begin{itemize}
        \item A) Variables that store memory addresses
        \item B) Variables that store the size of arrays
        \item C) Variables used only in dynamic memory allocation
        \item D) Variables that point to constants
    \end{itemize}

    \item What is wrong with this file-handling code?

    \lstset{language=C}
    \begin{lstlisting}
    FILE *fp = fopen("data.txt", "r");
    if (fp != NULL) {
        fprintf(fp, "Writing to file.");    
    }
    \end{lstlisting}
    
    \begin{itemize}
        \item A) File not opened in write mode
        \item B) File pointer not checked correctly
        \item C) Logical error in the condition
        \item D) Nothing is wrong
    \end{itemize}

    
\end{enumerate}

\newpage

\section{C++ Multiple Choice}

\begin{enumerate}

    \item Which is not a feature of OOP?
    \begin{itemize}
        \item A) Encapsulation
        \item B) Inheritance
        \item C) Polymorphism
        \item D) Procedural coding 
    \end{itemize}

    \item What is missing in the following class definition?

    \lstset{language=C}
    \begin{lstlisting}
    class Test {
    
        int x;
        void show();
    }
    \end{lstlisting}
    
    \begin{itemize}
        \item A) Missing semicolon after class declaration
        \item B) Function show is not defined
        \item C) Variable x not initialized
        \item D) Nothing is missing
    \end{itemize}

    \item Which of these best describes polymorphism?
    \begin{itemize}
        \item A) One name, multiple forms
        \item B) Ability to inherit properties
        \item C) Hiding data
        \item D) None of the above
    \end{itemize}

    \item What is wrong with the following destructor?

    \lstset{language=C}
    \begin{lstlisting}
    class Example {
    public:
        ~Example(int x) {
            cout << "Destructor called!";
        }
    };
    \end{lstlisting}
    
    \begin{itemize}
        \item A) Destructors cannot take arguments
        \item B) Destructor name is incorrect
        \item C) Destructor should not have a body
        \item D) Nothing is wrong
    \end{itemize}

    \item What does the \verb|new| operator do in C++?
    \begin{itemize}
        \item A) Allocates memory on the stack
        \item B) Allocates memory on the heap
        \item C) Allocates memory on both stack and heap
        \item D) Does not allocate memory
    \end{itemize}

\newpage

    \item What is the error in this code snippet?

    \lstset{language=C}
    \begin{lstlisting}
    int* ptr = new int[10];
    delete ptr;
    \end{lstlisting}
    
    \begin{itemize}
        \item A) Use of uninitialized pointer
        \item B) Memory is not freed properly
        \item C) Incorrect deletion of dynamically allocated array % Correct
        \item D) Nothing is wrong
    \end{itemize}

    \item What is the purpose of encapsulation in OOP?
    \begin{itemize}
        \item A) To reduce code size
        \item B) To hide implementation details and protect data % Correct
        \item C) To allow multiple inheritance
        \item D) To reuse existing classes
    \end{itemize}

    \item Which keyword is used for inheritance in C++?
    \begin{itemize}
        \item A) extends
        \item B) implements
        \item C) inherits
        \item D) public/private/protected % Correct
    \end{itemize}

    \item What are the two types of Abstraction?
    \begin{itemize}
        \item A) Data, Control
        \item B) Data, Restricted
        \item C) Control, Public
        \item D) Private, Public
    \end{itemize}

\begin{center}
    (Last question on the next page)
\end{center}

    \newpage

    \item Which class will be outputted by the following code?

    \lstset{language=C}
    \begin{lstlisting}
    class Base {
    public:
        virtual void show() {
        cout << "Base class";
        }
    };

    class Derived : public Base {
    public:
        void show() {
            cout << "Derived class";
        }
    };

    int main() {
        Base* b;
        Derived d;
        b = &d;
        b->show();
    }
    \end{lstlisting}

    \begin{itemize}
        \item A) Base class
        \item B) Derived class
        \item C) Compilation Error
        \item D) Undefined Behavior
    \end{itemize}
    
\end{enumerate}

\newpage

\section{C Creative Question}

\begin{enumerate}
    \item Define an integer array with the following specifications:\\

    The array should contain 10 elements: \verb|{12, 45, 78, 34, 23, 56, 89, 90, 67, 33}|.
    Declare this array in the main function and print all its elements to confirm the array initialization.\\

    (4 marks)\\

    \item Write a function named \verb|linearSearch| that performs the following tasks:\\

    Accept the \verb|array|, its \verb|size|, and a search value as arguments.\\
    Use a loop to iterate through the array and check if the search value exists.\\
    Return the \verb|index| of the element if it is found. If not, return -1.\\
    
    (8 marks)

    \item Modify the \verb|main| function to use the \verb|linearSearch| function:\\

    Prompt the user to input a value to search.\\
    Call the \verb|linearSearch| function and store the result.\\
    Print a message indicating whether the value was found and at what index.\\

    (8 marks)\\

    \item Write a function named \verb|findMaxAndMin| that performs the following:\\

    Accept the array and its size as arguments.\\
    Find and return both the maximum and minimum values in the array.\\
    Print these values in the main function after calling \verb|findMaxAndMin|.\\

    (6 marks)\\

    \item Write a function to calculate the sum and average of the array elements:\\

    The function should accept the array and its size as arguments.\\
    Calculate the sum and average of all elements in the array.\\
    Return these values to the main function and print them.\\

    (4 marks)\\

\end{enumerate}

\section{C++ Creative Question}

Be careful with the way you have formatted your answer, it should be easy to read.

\begin{enumerate}
    \item Define a Student class with the following specifications:\\

    Private member variables:\\
        \verb|name| (a string) to store the student's name.\\
        \verb|grades| (an array of 5 floats) to store the grades of 5 subjects.\\
        \verb|rollNumber| (an integer) to store the student's roll number.\\
        
    Public member functions:\\
        A constructor to initialise the name, grades, and rollNumber for each student.\\
        
    (5 marks)\\

    \item Write a member function named \verb|calculateAverage| in the \verb|Student| class. This function should:\\

        Compute the average of the grades stored in the \verb|grades| array.\\
        Return the calculated average as a \verb|float|.\\
        
    (5 marks)\\
    
    \item Add another member function named \verb|isPass| to the \verb|Student class|. This function should:\\

    Use the \verb|calculateAverage| function to determine if the student has passed.\\
        A student passes if their average grade is 50 or higher.\\
        Return \verb|true| if the student passes and \verb|false| otherwise.\\

    (5 marks)\\
    Write a member function named \verb|displayDetails| in the \verb|Student| class. This function should:\\

    Display the student's name, roll number, grades, average grade, and pass/fail status in a formatted manner. Here it is below:\\

    Name: John Doe\\
    Roll Number: 101\\
    Grades: 75.0 80.0 68.0 90.0 88.0\\
    Average Grade: 80.2\\
    Status: Pass\\

    \begin{center}
    Part 4 (final part) on the next page
    \end{center}

    \newpage
    
    \item In the main function, perform the following:

    Create two Student objects using the constructor. For example:\\
    
    Student 1: Name: "John Doe", Roll Number: 101, Grades: {75.0, 80.0, 68.0, 90.0, 88.0}\\
    Student 2: Name: "Jane Smith", Roll Number: 102, Grades: {40.0, 50.0, 45.0, 55.0, 60.0}\\

    Call the displayDetails function for both students to test all the implemented functionalities.\\

    (10 marks)
    
\end{enumerate}
\vfill
\end{document}
