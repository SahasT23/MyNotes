\documentclass[a4paper,12pt]{article}

%----------------------------------------------------------------------------------------
%	PACKAGES
%----------------------------------------------------------------------------------------
\usepackage{url}
\usepackage{parskip} 	
\usepackage{tabularx}
\RequirePackage{color}
\RequirePackage{graphicx}
\usepackage[usenames,dvipsnames]{xcolor}
\usepackage[scale=0.9]{geometry}
\usepackage{amsmath}
\usepackage{tikz}
\usetikzlibrary{shapes,arrows}
\usepackage{rotating}
\usetikzlibrary{shapes.geometric,arrows}
\usepackage{float}
\usepackage{algorithm}
\usepackage{algpseudocode}

\tikzstyle{startstop} = [rectangle, rounded corners, minimum width=3cm, minimum height=1cm, text centered, draw=black, fill=red!30]
\tikzstyle{process} = [rectangle, minimum width=3cm, minimum height=1cm, text centered, draw=black, fill=blue!30]
\tikzstyle{decision} = [diamond, minimum width=3cm, minimum height=1cm, text centered, draw=black, fill=green!30]
\tikzstyle{arrow} = [thick,->,>=stealth]

\definecolor{codegreen}{rgb}{0,0.6,0}
\definecolor{codegray}{rgb}{0.5,0.5,0.5}
\definecolor{codepurple}{rgb}{0.58,0,0.82}
\definecolor{backcolour}{rgb}{0.95,0.95,0.92}

\usepackage{listings}
\lstdefinestyle{mystyle}{
    backgroundcolor=\color{backcolour},   
    commentstyle=\color{codegreen},
    keywordstyle=\color{magenta},
    numberstyle=\tiny\color{codegray},
    stringstyle=\color{codepurple},
    basicstyle=\ttfamily\footnotesize,
    breakatwhitespace=false,         
    breaklines=true,                 
    captionpos=b,                    
    keepspaces=true,                 
    numbers=left,                    
    numbersep=5pt,                  
    showspaces=false,                
    showstringspaces=false,
    showtabs=false,                  
    tabsize=2
}
\lstset{style=mystyle}

%tabularx environment
\usepackage{tabularx}
\usepackage{enumitem}
\newcolumntype{C}{>{\centering\arraybackslash}X} 
\usepackage{supertabular}
\usepackage{tabularx}
\newlength{\fullcollw}
\setlength{\fullcollw}{0.47\textwidth}
\usepackage{titlesec}				
\usepackage{multicol}
\usepackage{multirow}
\titleformat{\section}{\large\scshape\raggedright}{}{0em}{}[\titlerule]
\titlespacing{\section}{0pt}{10pt}{10pt}
\usepackage[style=authoryear,sorting=ynt,maxbibnames=2]{biblatex}
\usepackage[unicode,draft=false]{hyperref}
\definecolor{linkcolour}{rgb}{0,0.2,0.6}
\hypersetup{colorlinks,breaklinks,urlcolor=linkcolour,linkcolor=linkcolour}
\addbibresource{citations.bib}
\setlength\bibitemsep{1em}
\usepackage{fontawesome5}

%----------------------------------------------------------------------------------------
%	BEGIN DOCUMENT
%----------------------------------------------------------------------------------------
\begin{document}

\title{C and C++ Programming Assessment 3}
\maketitle

\newpage
\pagestyle{empty}

\section{C Multiple Choice}

Each question will have four options, with each multiple choice question being worth \textbf{Two marks}.

\begin{enumerate}
    \item What is the size of a \verb|double| on most systems?
    \begin{itemize}
        \item A) 4 bytes
        \item B) 8 bytes
        \item C) 16 bytes
        \item D) 32 bytes
    \end{itemize}

    \item Which of the following is NOT a valid C data type?
    \begin{itemize}
        \item A) short
        \item B) long
        \item C) string
        \item D) char
    \end{itemize}

    \item What will this code output?

    \lstset{language=C}
    \begin{lstlisting}
        int x = 3;
        printf("%d", ++x + x++);
    \end{lstlisting}    
    
    \begin{itemize}
        \item A) 7
        \item B) 8
        \item C) 9
        \item D) Undefined Behavior
    \end{itemize}

    \item What is wrong with the following code?

    \lstset{language=C}
    \begin{lstlisting}
        int *ptr;
        *ptr = 10;
    \end{lstlisting}    
    
    \begin{itemize}
        \item A) Pointer not initialized
        \item B) Incorrect dereference syntax
        \item C) Invalid assignment
        \item D) Nothing is wrong
    \end{itemize}

    \item Which of the following is guaranteed to execute its condition check before the loop body?
    \begin{itemize}
        \item A) do-while
        \item B) while
        \item C) for
        \item D) Both B and C
    \end{itemize}

    \item What does the \verb|sizeof| operator return for an array passed to a function?
    \begin{itemize}
        \item A) Size of the entire array
        \item B) Size of the first element
        \item C) Size of a pointer
        \item D) Number of elements
    \end{itemize}

    \item What is wrong with the following code?

    \lstset{language=C}
    \begin{lstlisting}
        int arr[5] = {1, 2, 3};
        for (int i = 0; i <= 5; i++) {
            printf("%d ", arr[i]);
        }
    \end{lstlisting}

    \begin{itemize}
        \item A) Loop condition is incorrect
        \item B) Array initialization is invalid
        \item C) Missing format specifier
        \item D) Nothing is wrong
    \end{itemize}

    \item What is the purpose of the \verb|const| keyword in C?
    \begin{itemize}
        \item A) Defines a variable as immutable
        \item B) Allocates memory dynamically
        \item C) Restricts function scope
        \item D) Enables inline expansion
    \end{itemize}

    \item Which of the following declares a 2D array correctly?
    \begin{itemize}
        \item A) int arr[3,3];
        \item B) int arr[3][3];
        \item C) int arr(3)(3);
        \item D) int *arr[3][3];
    \end{itemize}

    \item What is wrong with this code?

    \lstset{language=C}
    \begin{lstlisting}
        FILE *fp = fopen("test.txt", "r");
        fscanf(fp, "%d");
    \end{lstlisting}
    
    \begin{itemize}
        \item A) Missing variable in fscanf
        \item B) File not checked for NULL
        \item C) Incorrect file mode
        \item D) Nothing is wrong
    \end{itemize}
\end{enumerate}

\newpage

\section{C++ Multiple Choice}

\begin{enumerate}
    \item Which of the following is a feature of C++ but not C?
    \begin{itemize}
        \item A) Pointers
        \item B) References
        \item C) Arrays
        \item D) Structures
    \end{itemize}

    \item What is wrong with this class definition?

    \lstset{language=C++}
    \begin{lstlisting}
        class Test {
            int data;
        public
            void setData(int d);
        };
    \end{lstlisting}
    
    \begin{itemize}
        \item A) Missing colon after public
        \item B) Missing semicolon after class
        \item C) Missing function implementation
        \item D) Nothing is wrong
    \end{itemize}

    \item What does the \verb|override| keyword ensure in C++?
    \begin{itemize}
        \item A) Function is virtual
        \item B) Function overrides a base class virtual function
        \item C) Function is static
        \item D) Function is inline
    \end{itemize}

    \item What is wrong with this code?

    \lstset{language=C++}
    \begin{lstlisting}
        class Example {
        public:
            Example() {
                int x = 0;
                return x;
            }
        };
    \end{lstlisting}
    
    \begin{itemize}
        \item A) Constructors cannot declare variables
        \item B) Constructors cannot return values
        \item C) Constructor name is incorrect
        \item D) Nothing is wrong
    \end{itemize}

    \item What does the \verb|this| pointer refer to in a C++ class?
    \begin{itemize}
        \item A) Current class definition
        \item B) Current object instance
        \item C) Base class instance
        \item D) Static member
    \end{itemize}

    \item What is the error in this code?

    \lstset{language=C++}
    \begin{lstlisting}
        int *ptr = new int;
        delete[] ptr;
    \end{lstlisting}
    
    \begin{itemize}
        \item A) Incorrect deletion syntax for single object
        \item B) Memory leak
        \item C) Pointer not initialized
        \item D) Nothing is wrong
    \end{itemize}

    \item What is the purpose of operator overloading in C++?
    \begin{itemize}
        \item A) To redefine operators for user-defined types
        \item B) To prevent operator usage
        \item C) To restrict operator scope
        \item D) To inline operator functions
    \end{itemize}

    \item Which keyword prevents a class from being inherited?
    \begin{itemize}
        \item A) sealed
        \item B) final
        \item C) private
        \item D) static
    \end{itemize}

    \item What is the difference between \verb|struct| and \verb|class| in C++?
    \begin{itemize}
        \item A) Default access specifier
        \item B) Memory allocation method
        \item C) Inheritance support
        \item D) Function definition scope
    \end{itemize}

    \begin{center}
        (Last question on the next page)
    \end{center}

    \newpage

    \item What will be the output of this code?

    \lstset{language=C++}
    \begin{lstlisting}
        class Base {
        public:
            virtual void print() {
                cout << "Base";
            }
        };
        class Derived : public Base {
        public:
            void print() override {
                cout << "Derived";
            }
        };
        int main() {
            Base *b = new Derived();
            b->print();
            delete b;
        }
    \end{lstlisting}

    \begin{itemize}
        \item A) Base
        \item B) Derived
        \item C) Compilation Error
        \item D) Undefined Behavior
    \end{itemize}
\end{enumerate}

\newpage

\section{C Creative Question}

C Programming Topic: Array Reversal

\begin{enumerate}
    \item Define an integer array with the following specifications:

    The array should contain 8 elements: \verb|{5, 10, 15, 20, 25, 30, 35, 40}|.
    Declare this array in the main function and print all its elements to confirm initialization.

    (4 marks)

    \item Write a function named \verb|reverseArray| that performs the following tasks:

    Accept the \verb|array| and its \verb|size| as arguments.
    Reverse the array in place using a loop.
    Return void, as the array is modified directly.

    (8 marks)

    \item Modify the \verb|main| function to use the \verb|reverseArray| function:

    Call the \verb|reverseArray| function.
    Print the array elements after reversal to confirm the operation.

    (8 marks)

    \item Write a function named \verb|findDuplicates| that performs the following:

    Accept the array and its size as arguments.
    Check for duplicate elements in the array.
    Print any duplicates found in the main function after calling \verb|findDuplicates|.

    (6 marks)

    \item Write a function to calculate the range of the array elements:

    Accept the array and its size as arguments.
    Calculate the range (difference between max and min elements).
    Return the range and print it in the main function.

    (4 marks)
\end{enumerate}

\section{C++ Creative Question}

Be careful with the formatting of your answer; it should be easy to read.

\begin{enumerate}
    \item Define a \verb|Book| class with the following specifications:

    Private member variables:
        \verb|title| (a string) to store the book’s title.
        \verb|ratings| (an array of 4 floats) to store reader ratings.
        \verb|isbn| (an integer) to store the book’s ISBN number.

    Public member functions:
        A constructor to initialize the title, ratings, and ISBN for each book.

    (5 marks)

    \item Write a member function named \verb|calculateAverageRating| in the \verb|Book| class that:

    Computes the average of the ratings stored in the \verb|ratings| array.
    Returns the calculated average as a \verb|float|.

    (5 marks)

    \item Add a member function named \verb|isHighlyRated| to the \verb|Book| class that:

    Uses \verb|calculateAverageRating| to check if the average rating is 4.0 or higher.
    Returns \verb|true| if the average rating is 4.0 or higher, \verb|false| otherwise.

    (5 marks)

    \item Write a member function named \verb|displayBookInfo| in the \verb|Book| class that:

    Displays the book’s title, ISBN, ratings, average rating, and rating status in the following format:

    Title: The Great Gatsby\\
    ISBN: 123456789\\
    Ratings: 4.5 4.0 3.8 4.2\\
    Average Rating: 4.13\\
    Status: Highly Rated\\

    \begin{center}
        Part 4 (final part) on the next page
    \end{center}

    \newpage

    \item In the main function, perform the following:

    Create two \verb|Book| objects using the constructor. For example:
    
    Book 1: Title: "The Great Gatsby", ISBN: 123456789, Ratings: \{4.5, 4.0, 3.8, 4.2\}\\
    Book 2: Title: "1984", ISBN: 987654321, Ratings: \{3.5, 3.0, 3.2, 3.8\}\\

    Call the \verb|displayBookInfo| function for both books to test all functionalities.

    (10 marks)
\end{enumerate}
\vfill
\end{document}