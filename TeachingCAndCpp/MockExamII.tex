\documentclass[a4paper,12pt]{article}

%----------------------------------------------------------------------------------------
%	PACKAGES
%----------------------------------------------------------------------------------------
\usepackage{url}
\usepackage{parskip} 	
\usepackage{tabularx}
\RequirePackage{color}
\RequirePackage{graphicx}
\usepackage[usenames,dvipsnames]{xcolor}
\usepackage[scale=0.9]{geometry}
\usepackage{amsmath}
\usepackage{tikz}
\usetikzlibrary{shapes,arrows}
\usepackage{rotating}
\usetikzlibrary{shapes.geometric,arrows}
\usepackage{float}
\usepackage{algorithm}
\usepackage{algpseudocode}

\tikzstyle{startstop} = [rectangle, rounded corners, minimum width=3cm, minimum height=1cm, text centered, draw=black, fill=red!30]
\tikzstyle{process} = [rectangle, minimum width=3cm, minimum height=1cm, text centered, draw=black, fill=blue!30]
\tikzstyle{decision} = [diamond, minimum width=3cm, minimum height=1cm, text centered, draw=black, fill=green!30]
\tikzstyle{arrow} = [thick,->,>=stealth]

\definecolor{codegreen}{rgb}{0,0.6,0}
\definecolor{codegray}{rgb}{0.5,0.5,0.5}
\definecolor{codepurple}{rgb}{0.58,0,0.82}
\definecolor{backcolour}{rgb}{0.95,0.95,0.92}

\usepackage{listings}
\lstdefinestyle{mystyle}{
    backgroundcolor=\color{backcolour},   
    commentstyle=\color{codegreen},
    keywordstyle=\color{magenta},
    numberstyle=\tiny\color{codegray},
    stringstyle=\color{codepurple},
    basicstyle=\ttfamily\footnotesize,
    breakatwhitespace=false,         
    breaklines=true,                 
    captionpos=b,                    
    keepspaces=true,                 
    numbers=left,                    
    numbersep=5pt,                  
    showspaces=false,                
    showstringspaces=false,
    showtabs=false,                  
    tabsize=2
}
\lstset{style=mystyle}

%tabularx environment
\usepackage{tabularx}
\usepackage{enumitem}
\newcolumntype{C}{>{\centering\arraybackslash}X} 
\usepackage{supertabular}
\usepackage{tabularx}
\newlength{\fullcollw}
\setlength{\fullcollw}{0.47\textwidth}
\usepackage{titlesec}				
\usepackage{multicol}
\usepackage{multirow}
\titleformat{\section}{\large\scshape\raggedright}{}{0em}{}[\titlerule]
\titlespacing{\section}{0pt}{10pt}{10pt}
\usepackage[style=authoryear,sorting=ynt,maxbibnames=2]{biblatex}
\usepackage[unicode,draft=false]{hyperref}
\definecolor{linkcolour}{rgb}{0,0.2,0.6}
\hypersetup{colorlinks,breaklinks,urlcolor=linkcolour,linkcolor=linkcolour}
\addbibresource{citations.bib}
\setlength\bibitemsep{1em}
\usepackage{fontawesome5}

%----------------------------------------------------------------------------------------
%	BEGIN DOCUMENT
%----------------------------------------------------------------------------------------
\begin{document}

\title{C and C++ Programming Assessment 2}
\maketitle

\newpage
\pagestyle{empty}

\section{C Multiple Choice}

Each question will have four options, with each multiple choice question being worth \textbf{Two marks}.

\begin{enumerate}
    \item What is the size of a \verb|float| on most systems?
    \begin{itemize}
        \item A) 2 bytes
        \item B) 4 bytes
        \item C) 8 bytes
        \item D) 16 bytes
    \end{itemize}

    \item Which of the following is a valid variable name in C?
    \begin{itemize}
        \item A) \verb|1variable|
        \item B) \verb|variable_1|
        \item C) \verb|variable-1|
        \item D) \verb|variable#1|
    \end{itemize}

    \item What will this code output?

    \lstset{language=C}
    \begin{lstlisting}
        int x = 5;
        printf("%d", x++ * 2);
    \end{lstlisting}    
    
    \begin{itemize}
        \item A) 10
        \item B) 12
        \item C) 5
        \item D) Undefined Behavior
    \end{itemize}

    \item What is wrong with the following code?

    \lstset{language=C}
    \begin{lstlisting}
        char str[10];
        strcpy(str, "Hello World");
    \end{lstlisting}    
    
    \begin{itemize}
        \item A) Missing include directive
        \item B) Buffer overflow
        \item C) Incorrect use of strcpy
        \item D) Nothing is wrong
    \end{itemize}

    \item Which loop is best suited for iterating over an array of known size?
    \begin{itemize}
        \item A) while
        \item B) do-while
        \item C) for
        \item D) None of the above
    \end{itemize}

    \item Which of the following cannot be used to initialize an array in C?
    \begin{itemize}
        \item A) Static initialization
        \item B) Dynamic initialization
        \item C) Direct assignment of another array
        \item D) Element-by-element assignment
    \end{itemize}

    \item What is wrong with the following code?

    \lstset{language=C}
    \begin{lstlisting}
        int arr[3] = {1, 2, 3};
        int *ptr = arr;
        printf("%d", *(ptr + 3));
    \end{lstlisting}

    \begin{itemize}
        \item A) Pointer arithmetic is incorrect
        \item B) Out-of-bounds access
        \item C) Array initialization is wrong
        \item D) Nothing is wrong
    \end{itemize}

    \item What is the purpose of the \verb|static| keyword in a function?
    \begin{itemize}
        \item A) Restricts variable scope to the file
        \item B) Preserves variable value between function calls
        \item C) Allocates memory dynamically
        \item D) Makes the function globally accessible
    \end{itemize}

    \item Which of the following correctly declares a pointer to a function?
    \begin{itemize}
        \item A) int *func();
        \item B) int (*func)();
        \item C) int *func();
        \item D) int func*();
    \end{itemize}

    \item What is wrong with this code?

    \lstset{language=C}
    \begin{lstlisting}
        FILE *fp = fopen("data.txt", "w");
        fclose(fp);
        fprintf(fp, "Test");
    \end{lstlisting}
    
    \begin{itemize}
        \item A) File opened in wrong mode
        \item B) Writing to a closed file
        \item C) Missing error checking
        \item D) Nothing is wrong
    \end{itemize}
\end{enumerate}

\newpage

\section{C++ Multiple Choice}

\begin{enumerate}
    \item Which of the following is NOT a C++ access specifier?
    \begin{itemize}
        \item A) public
        \item B) private
        \item C) protected
        \item D) internal
    \end{itemize}

    \item What is wrong with this class definition?

    \lstset{language=C++}
    \begin{lstlisting}
        class Demo {
            int value
            void setValue(int v);
        };
    \end{lstlisting}
    
    \begin{itemize}
        \item A) Missing semicolon after class
        \item B) Missing access specifier
        \item C) Function not implemented
        \item D) Nothing is wrong
    \end{itemize}

    \item What does the \verb|virtual| keyword enable in C++?
    \begin{itemize}
        \item A) Static binding
        \item B) Dynamic binding
        \item C) Data hiding
        \item D) Encapsulation
    \end{itemize}

    \item What is wrong with this constructor?

    \lstset{language=C++}
    \begin{lstlisting}
        class Sample {
        public:
            Sample(int x) {
                return x;
            }
        };
    \end{lstlisting}
    
    \begin{itemize}
        \item A) Constructors cannot have parameters
        \item B) Constructors cannot return values
        \item C) Incorrect access specifier
        \item D) Nothing is wrong
    \end{itemize}

    \item What is the purpose of the \verb|delete| operator in C++?
    \begin{itemize}
        \item A) Deallocates memory on the stack
        \item B) Deallocates memory on the heap
        \item C) Deletes a class definition
        \item D) Deletes a variable
    \end{itemize}

    \item What is the error in this code?

    \lstset{language=C++}
    \begin{lstlisting}
        int* arr = new int[5];
        arr[5] = 10;
    \end{lstlisting}
    
    \begin{itemize}
        \item A) Out-of-bounds access
        \item B) Memory leak
        \item C) Incorrect deletion syntax
        \item D) Nothing is wrong
    \end{itemize}

    \item What is the role of a copy constructor in C++?
    \begin{itemize}
        \item A) Initializes an object with default values
        \item B) Creates a copy of an existing object
        \item C) Deletes an object
        \item D) Allocates memory dynamically
    \end{itemize}

    \item Which keyword is used to define a constant member function?
    \begin{itemize}
        \item A) static
        \item B) const
        \item C) final
        \item D) virtual
    \end{itemize}

    \item What are the types of inheritance in C++?
    \begin{itemize}
        \item A) Single, Multiple, Multilevel, Hierarchical, Hybrid
        \item B) Public, Private, Protected
        \item C) Static, Dynamic
        \item D) Abstract, Concrete
    \end{itemize}

    \begin{center}
        (Last question on the next page)
    \end{center}

    \newpage

    \item What will be the output of this code?

    \lstset{language=C++}
    \begin{lstlisting}
        class A {
        public:
            virtual void display() {
                cout << "Class A";
            }
        };
        class B : public A {
        public:
            void display() {
                cout << "Class B";
            }
        };
        int main() {
            A *a = new B();
            a->display();
            delete a;
        }
    \end{lstlisting}

    \begin{itemize}
        \item A) Class A
        \item B) Class B
        \item C) Compilation Error
        \item D) Undefined Behavior
    \end{itemize}
\end{enumerate}

\newpage

\section{C Creative Question}

C Programming Topic: Binary Search

\begin{enumerate}
    \item Define an integer array with the following specifications:

    The array should contain 10 elements: \verb|{10, 20, 30, 40, 50, 60, 70, 80, 90, 100}|, sorted in ascending order.
    Declare this array in the main function and print all its elements to confirm initialization.

    (4 marks)

    \item Write a function named \verb|binarySearch| that performs the following tasks:

    Accept the sorted \verb|array|, its \verb|size|, and a search value as arguments.
    Implement binary search to find the index of the search value.
    Return the \verb|index| if found; otherwise, return -1.

    (8 marks)

    \item Modify the \verb|main| function to use the \verb|binarySearch| function:

    Prompt the user to input a value to search.
    Call the \verb|binarySearch| function and store the result.
    Print whether the value was found and its index.

    (8 marks)

    \item Write a function named \verb|findSecondLargest| that performs the following:

    Accept the array and its size as arguments.
    Find and return the second largest value in the array.
    Print this value in the main function after calling \verb|findSecondLargest|.

    (6 marks)

    \item Write a function to calculate the median of the array elements:

    Accept the array and its size as arguments.
    Calculate the median (average of two middle elements for even-sized array).
    Return the median and print it in the main function.

    (4 marks)
\end{enumerate}

\section{C++ Creative Question}

Be careful with the formatting of your answer; it should be easy to read.

\begin{enumerate}
    \item Define an \verb|Employee| class with the following specifications:

    Private member variables:
        \verb|name| (a string) to store the employee's name.
        \verb|salaries| (an array of 6 floats) to store monthly salaries for half a year.
        \verb|id| (an integer) to store the employee's ID.

    Public member functions:
        A constructor to initialize the name, salaries, and ID for each employee.

    (5 marks)

    \item Write a member function named \verb|calculateAverageSalary| in the \verb|Employee| class that:

    Computes the average of the salaries stored in the \verb|salaries| array.
    Returns the calculated average as a \verb|float|.

    (5 marks)

    \item Add a member function named \verb|isAboveThreshold| to the \verb|Employee| class that:

    Uses \verb|calculateAverageSalary| to check if the average salary exceeds 5000.
    Returns \verb|true| if the average salary is above 5000, \verb|false| otherwise.

    (5 marks)

    \item Write a member function named \verb|displayInfo| in the \verb|Employee| class that:

    Displays the employee's name, ID, salaries, average salary, and threshold status in the following format:

    Name: Alice Johnson\\
    ID: 201\\
    Salaries: 4500.0 4700.0 4800.0 4900.0 5000.0 5100.0\\
    Average Salary: 4833.33\\
    Status: Below Threshold\\

    \begin{center}
        Part 4 (final part) on the next page
    \end{center}

    \newpage

    \item In the main function, perform the following:

    Create two \verb|Employee| objects using the constructor. For example:
    
    Employee 1: Name: "Alice Johnson", ID: 201, Salaries: \{4500.0, 4700.0, 4800.0, 4900.0, 5000.0, 5100.0\}\\
    Employee 2: Name: "Bob Wilson", ID: 202, Salaries: \{5500.0, 5600.0, 5700.0, 5800.0, 5900.0, 6000.0\}\\

    Call the \verb|displayInfo| function for both employees to test all functionalities.

    (10 marks)
\end{enumerate}
\vfill
\end{document}