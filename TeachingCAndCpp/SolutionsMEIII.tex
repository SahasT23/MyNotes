\documentclass[a4paper,12pt]{article}

%----------------------------------------------------------------------------------------
%	PACKAGES
%----------------------------------------------------------------------------------------
\usepackage{url}
\usepackage{parskip} 	
\usepackage{tabularx}
\RequirePackage{color}
\RequirePackage{graphicx}
\usepackage[usenames,dvipsnames]{xcolor}
\usepackage[scale=0.9]{geometry}
\usepackage{amsmath}
\usepackage{tikz}
\usetikzlibrary{shapes,arrows}
\usepackage{rotating}
\usetikzlibrary{shapes.geometric,arrows}
\usepackage{float}
\usepackage{algorithm}
\usepackage{algpseudocode}

\definecolor{codegreen}{rgb}{0,0.6,0}
\definecolor{codegray}{rgb}{0.5,0.5,0.5}
\definecolor{codepurple}{rgb}{0.58,0,0.82}
\definecolor{backcolour}{rgb}{0.95,0.95,0.92}

\usepackage{listings}
\lstdefinestyle{mystyle}{
    backgroundcolor=\color{backcolour},   
    commentstyle=\color{codegreen},
    keywordstyle=\color{magenta},
    numberstyle=\tiny\color{codegray},
    stringstyle=\color{codepurple},
    basicstyle=\ttfamily\footnotesize,
    breakatwhitespace=false,         
    breaklines=true,                 
    captionpos=b,                    
    keepspaces=true,                 
    numbers=left,                    
    numbersep=5pt,                  
    showspaces=false,                
    showstringspaces=false,
    showtabs=false,                  
    tabsize=2
}
\lstset{style=mystyle}

\usepackage{tabularx}
\usepackage{enumitem}
\newcolumntype{C}{>{\centering\arraybackslash}X} 
\usepackage{supertabular}
\usepackage{tabularx}
\newlength{\fullcollw}
\setlength{\fullcollw}{0.47\textwidth}
\usepackage{titlesec}				
\usepackage{multicol}
\usepackage{multirow}
\titleformat{\section}{\large\scshape\raggedright}{}{0em}{}[\titlerule]
\titlespacing{\section}{0pt}{10pt}{10pt}
\usepackage[style=authoryear,sorting=ynt,maxbibnames=2]{biblatex}
\usepackage[unicode,draft=false]{hyperref}
\definecolor{linkcolour}{rgb}{0,0.2,0.6}
\hypersetup{colorlinks,breaklinks,urlcolor=linkcolour,linkcolor=linkcolour}
\addbibresource{citations.bib}
\setlength\bibitemsep{1em}
\usepackage{fontawesome5}

%----------------------------------------------------------------------------------------
%	BEGIN DOCUMENT
%----------------------------------------------------------------------------------------
\begin{document}

\title{C and C++ Programming Assessment 3 Answers}
\maketitle

\newpage
\pagestyle{empty}

\section{C Multiple Choice Answers}

\begin{enumerate}
    \item \textbf{B) 8 bytes}

    A \verb|double| typically occupies 8 bytes on most systems, following the IEEE 754 double-precision format.

    \item \textbf{C) string}

    C does not have a built-in \verb|string| data type; strings are handled as arrays of \verb|char|. Other options are valid C types.

    \item \textbf{D) Undefined Behavior}

    The expression \verb|++x + x++| modifies \verb|x| multiple times without a sequence point, leading to undefined behavior.

    \item \textbf{A) Pointer not initialized}

    The pointer \verb|ptr| is not initialized before dereferencing, causing undefined behavior when assigning \verb|*ptr = 10|.

    \item \textbf{D) Both B and C}

    Both \verb|while| and \verb|for| loops check their condition before executing the body, unlike \verb|do-while|.

    \item \textbf{C) Size of a pointer}

    When an array is passed to a function, it decays to a pointer, so \verb|sizeof| returns the size of the pointer (typically 4 or 8 bytes).

    \item \textbf{A) Loop condition is incorrect}

    The loop condition \verb|i <= 5| accesses \verb|arr[5]|, which is out of bounds for \verb|arr[5]| (indices 0 to 4).

    \item \textbf{A) Defines a variable as immutable}

    The \verb|const| keyword declares a variable as read-only, preventing modification after initialization.

    \item \textbf{B) int arr[3][3];}

    A 2D array is declared as \verb|int arr[rows][cols]|. Option A uses incorrect syntax, C is invalid, and D declares an array of pointers.

    \item \textbf{A) Missing variable in fscanf}

    \verb|fscanf| requires a variable to store the scanned value (e.g., \verb|fscanf(fp, "%d", &x)|), otherwise it causes undefined behavior.
\end{enumerate}

\section{C++ Multiple Choice Answers}

\begin{enumerate}
    \item \textbf{B) References}

    References are a C++ feature, not available in C. Pointers, arrays, and structures exist in both languages.

    \item \textbf{B) Missing semicolon after class}

    A class definition requires a semicolon after the closing brace. The syntax for \verb|public| is correct but lacks the semicolon.

    \item \textbf{B) Function overrides a base class virtual function}

    The \verb|override| keyword ensures a function overrides a virtual function in the base class, preventing errors.

    \item \textbf{B) Constructors cannot return values}

    Constructors initialize objects and cannot return values, even local variables like \verb|x|.

    \item \textbf{B) Current object instance}

    The \verb|this| pointer refers to the current object instance within a class’s member functions.

    \item \textbf{A) Incorrect deletion syntax for single object}

    A single object allocated with \verb|new| should be deleted with \verb|delete|, not \verb|delete[]|, which is for arrays.

    \item \textbf{A) To redefine operators for user-defined types}

    Operator overloading allows custom behavior for operators (e.g., \verb|+|) with user-defined classes.

    \item \textbf{B) final}

    The \verb|final| keyword prevents a class from being inherited or a virtual function from being overridden.

    \item \textbf{A) Default access specifier}

    In a \verb|struct|, members are \verb|public| by default; in a \verb|class|, they are \verb|private|.

    \item \textbf{B) Derived}

    The \verb|virtual| function enables dynamic binding, so the \verb|print| function of \verb|Derived| is called, outputting "Derived".
\end{enumerate}

\section{C Creative Question Answers}

\begin{enumerate}
    \item \textbf{Array Definition and Printing}

    \lstset{language=C}
    \begin{lstlisting}
        int main() {
            int arr[8] = {5, 10, 15, 20, 25, 30, 35, 40};
            for (int i = 0; i < 8; i++) {
                printf("%d ", arr[i]);
            }
            printf("\n");
            return 0;
        }
    \end{lstlisting}

    Initializes and prints the array to confirm.

    \item \textbf{Reverse Array Function}

    \lstset{language=C}
    \begin{lstlisting}
        void reverseArray(int arr[], int size) {
            for (int i = 0; i < size / 2; i++) {
                int temp = arr[i];
                arr[i] = arr[size - 1 - i];
                arr[size - 1 - i] = temp;
            }
        }
    \end{lstlisting}

    Reverses the array in place by swapping elements from the ends.

    \item \textbf{Modify main with Reverse Array}

    \lstset{language=C}
    \begin{lstlisting}
        int main() {
            int arr[8] = {5, 10, 15, 20, 25, 30, 35, 40};
            printf("Original array: ");
            for (int i = 0; i < 8; i++) {
                printf("%d ", arr[i]);
            }
            printf("\n");
            reverseArray(arr, 8);
            printf("Reversed array: ");
            for (int i = 0; i < 8; i++) {
                printf("%d ", arr[i]);
            }
            printf("\n");
            return 0;
        }
    \end{lstlisting}

    Calls \verb|reverseArray| and prints the reversed array.

    \item \textbf{Find Duplicates}

    \lstset{language=C}
    \begin{lstlisting}
        void findDuplicates(int arr[], int size) {
            int found = 0;
            for (int i = 0; i < size; i++) {
                for (int j = i + 1; j < size; j++) {
                    if (arr[i] == arr[j]) {
                        printf("Duplicate found: %d\n", arr[i]);
                        found = 1;
                    }
                }
            }
            if (!found) printf("No duplicates found\n");
        }
    \end{lstlisting}

    Checks for duplicates and prints them; called in \verb|main|.

    \item \textbf{Calculate Range}

    \lstset{language=C}
    \begin{lstlisting}
        int calculateRange(int arr[], int size) {
            int max = arr[0], min = arr[0];
            for (int i = 1; i < size; i++) {
                if (arr[i] > max) max = arr[i];
                if (arr[i] < min) min = arr[i];
            }
            return max - min;
        }
    \end{lstlisting}

    Returns the range (max - min), printed in \verb|main|.
\end{enumerate}

\section{C++ Creative Question Answers}

\begin{enumerate}
    \item \textbf{Book Class Definition}

    \lstset{language=C++}
    \begin{lstlisting}
        class Book {
        private:
            string title;
            float ratings[4];
            int isbn;
        public:
            Book(string t, float r[], int i) {
                title = t;
                for (int j = 0; j < 4; j++) ratings[j] = r[j];
                isbn = i;
            }
        };
    \end{lstlisting}

    Defines the class with private members and a constructor.

    \item \textbf{Calculate Average Rating}

    \lstset{language=C++}
    \begin{lstlisting}
        float calculateAverageRating() {
            float sum = 0;
            for (int i = 0; i < 4; i++) {
                sum += ratings[i];
            }
            return sum / 4;
        }
    \end{lstlisting}

    Computes and returns the average rating.

    \item \textbf{Is Highly Rated}

    \lstset{language=C++}
    \begin{lstlisting}
        bool isHighlyRated() {
            return calculateAverageRating() >= 4.0;
        }
    \end{lstlisting}

    Returns \verb|true| if the average rating is 4.0 or higher.

    \item \textbf{Display Book Info}

    \lstset{language=C++}
    \begin{lstlisting}
        void displayBookInfo() {
            cout << "Title: " << title << endl;
            cout << "ISBN: " << isbn << endl;
            cout << "Ratings: ";
            for (int i = 0; i < 4; i++) {
                cout << ratings[i] << " ";
            }
            cout << endl;
            float avg = calculateAverageRating();
            cout << "Average Rating: " << fixed << setprecision(2) << avg << endl;
            cout << "Status: " << (isHighlyRated() ? "Highly Rated" : "Not Highly Rated") << endl;
        }
    \end{lstlisting}

    Displays book details in the specified format.

    \item \textbf{Main Function}

    \lstset{language=C++}
    \begin{lstlisting}
        int main() {
            float ratings1[4] = {4.5, 4.0, 3.8, 4.2};
            float ratings2[4] = {3.5, 3.0, 3.2, 3.8};
            Book book1("The Great Gatsby", ratings1, 123456789);
            Book book2("1984", ratings2, 987654321);
            book1.displayBookInfo();
            cout << endl;
            book2.displayBookInfo();
            return 0;
        }
    \end{lstlisting}

    Creates two \verb|Book| objects and tests all functionalities.
\end{enumerate}
\end{document}