% !TeX root = main.tex

\documentclass[a4paper,12pt]{article}

%----------------------------------------------------------------------------------------
%	FONT
%----------------------------------------------------------------------------------------

%----------------------------------------------------------------------------------------
%	PACKAGES
%----------------------------------------------------------------------------------------
\usepackage{url}
\usepackage{parskip} 	
\usepackage{tabularx}

%other packages for formatting
\RequirePackage{color}
\RequirePackage{graphicx}
\usepackage[usenames,dvipsnames]{xcolor}
\usepackage[scale=0.9]{geometry}
\usepackage{amsmath}
\usepackage{tikz}
\usetikzlibrary{shapes,arrows}
\usepackage{rotating}
\usetikzlibrary{shapes.geometric, arrows}
\usepackage{float}
\usepackage{algorithm}
\usepackage{algpseudocode}
\usetikzlibrary{positioning, calc}

\usepackage{float}
\usepackage{algorithm}
\usepackage{algpseudocode}
\usepackage{tikz}
\usetikzlibrary{arrows.meta, positioning}
\usepackage{setspace}
\usepackage{titlesec}
\usepackage{tikz}
\usepackage{tabularx}
\usepackage{circuitikz}
\usepackage{pgfplots}
\usepackage{amssymb}
\usepackage{amsthm}
\usepackage{minted}
\usepackage{float}  
\usepackage{booktabs}
\usepackage{xcolor}
\usepackage{longtable}
\pgfplotsset{compat=1.18}
\usepackage{setspace}
\theoremstyle{definition}
\newtheorem{definition}{Definition}
\usepackage{amsthm}
\usepackage{adjustbox}

\tikzstyle{startstop} = [rectangle, rounded corners, minimum width=3cm, minimum height=1cm,text centered, draw=black, fill=red!30]
\tikzstyle{process} = [rectangle, minimum width=3cm, minimum height=1cm, text centered, draw=black, fill=blue!30]
\tikzstyle{decision} = [diamond, minimum width=3cm, minimum height=1cm, text centered, draw=black, fill=green!30]
\tikzstyle{arrow} = [thick,->,>=stealth]

\definecolor{codegreen}{rgb}{0,0.6,0}
\definecolor{codegray}{rgb}{0.5,0.5,0.5}
\definecolor{codepurple}{rgb}{0.58,0,0.82}
\definecolor{backcolour}{rgb}{0.95,0.95,0.92}

\usepackage{listings}

\lstdefinestyle{mystyle}{
    backgroundcolor=\color{backcolour},   
    commentstyle=\color{codegreen},
    keywordstyle=\color{magenta},
    numberstyle=\tiny\color{codegray},
    stringstyle=\color{codepurple},
    basicstyle=\ttfamily\footnotesize,
    breakatwhitespace=false,         
    breaklines=true,                 
    captionpos=b,                    
    keepspaces=true,                 
    numbers=left,                    
    numbersep=5pt,                  
    showspaces=false,                
    showstringspaces=false,
    showtabs=false,                  
    tabsize=2
}

\lstset{style=mystyle}

%tabularx environment
\usepackage{tabularx}

%for lists within experience section
\usepackage{enumitem}

% centered version of 'X' col. type
\newcolumntype{C}{>{\centering\arraybackslash}X} 

%to prevent spillover of tabular into next pages
\usepackage{supertabular}
\usepackage{tabularx}
\newlength{\fullcollw}
\setlength{\fullcollw}{0.47\textwidth}

%custom \section
\usepackage{titlesec}				
\usepackage{multicol}
\usepackage{multirow}

%CV Sections inspired by: 
%http://stefano.italians.nl/archives/26
\titleformat{\section}{\large\scshape\raggedright}{}{0em}{}[\titlerule]
\titlespacing{\section}{0pt}{10pt}{10pt}

%for publications
\usepackage[style=authoryear,sorting=ynt, maxbibnames=2]{biblatex}

%Setup hyperref package, and colours for links
\usepackage[unicode, draft=false]{hyperref}
\definecolor{linkcolour}{rgb}{0,0.2,0.6}
\hypersetup{colorlinks,breaklinks,urlcolor=linkcolour,linkcolor=linkcolour}
\addbibresource{citations.bib}
\setlength\bibitemsep{1em}

%for social icons
\usepackage{fontawesome5}


% Remove the date
\date{}

% Title format

\begin{document}

\onehalfspacing%
% Title Page
\begin{titlepage}
    \centering
    \vspace*{2cm}
    \Huge{\textbf{Git, Python and ML Shortcuts for internship}} \\[1.5cm]
    \Large{\textit{230057896}\\
    Sahas Talasila} \\[1cm]
    \vfill
\end{titlepage}

% Table of Contents
\tableofcontents
\newpage

\section{Git}

1. Start by using \verb|mkdir| command to create the project folder.

2.\ then use:

\begin{lstlisting}
    cd /path/to/your/project
\end{lstlisting}

3. Then initialise using the command below so that git will actually track the changes in your project folder:

\begin{lstlisting}
    git init
\end{lstlisting}

4. Then add a `README.md' markdown file which will allow for a user to understand how the project actually works:

\begin{lstlisting}
    touch README.md
\end{lstlisting}

5. Add a `.gitignore' file which will prevent other users from getting access to certain files that they should not have access to (API keys, certain personal details)

6. Then add to the git tracking:

\begin{lstlisting}
    git add .
\end{lstlisting}

7. Then save and commit the changes with a descriptive commit message:

\begin{lstlisting}
    git commit -m ''Initial commit''
\end{lstlisting}

8. Log in to GitHub, go to GitHub, and create a new repository called TestProject. Make sure to select ``Private'' so that only you can access it.

9. Link your local project to GitHub
Once the repository is created, copy its URL and add it to your local repository:

\begin{lstlisting}
    git remote add origin https://github.com/YOUR_USERNAME/TestProject.git
\end{lstlisting}

10. Then for simplicity, rename to `main'.

\begin{lstlisting}
    git branch -M main
\end{lstlisting}

11. Then, finally push to the main branch and push to the remote repository on Github

\begin{lstlisting}
    git push -u origin main
\end{lstlisting}

\begin{lstlisting}
    git pull origin main --rebase
\end{lstlisting}

for moving from one directory to another

\begin{lstlisting}
    mv
\end{lstlisting}

Follow these steps to clone a repository:

\begin{enumerate}
    \item Navigate to the GitHub repository page.
    \item Click on the green \textbf{Code} button.
    \item Copy the repository URL (HTTPS or SSH).
    \item Open a terminal and run the following command:
\end{enumerate}

\begin{lstlisting}[language=bash]
git clone https://github.com/username/repository.git
\end{lstlisting}

Replace \texttt{username} and \texttt{repository} with the actual repository owner and name.

Navigating and Using the Repository
Once cloned, move into the repository directory:

\begin{lstlisting}[language=bash]
cd repository
\end{lstlisting}

You can now edit files, create branches, and push changes.

\subsection{Cloning}
Cloning a repository is the first step in working on a GitHub project locally. After cloning, you can modify code, commit changes, and sync them with the remote repository.

For more details, visit \href{https://git-scm.com/}{Git Documentation}.

\subsection{Git Branching}

\textbf{Creating a new branch:}
\begin{verbatim}
git branch <branch-name>
\end{verbatim}
This command creates a new branch, allowing you to work on features separately from the main branch.

\textbf{Listing all branches:}
\begin{lstlisting}
    git branch
\end{lstlisting}

Use this command to view all branches in your repository.

\textbf{Switching to an existing branch:}
\begin{lstlisting}
git checkout <branch-name>
\end{lstlisting}

Alternatively, with newer Git versions:
\begin{lstlisting}
git switch <branch-name>
\end{lstlisting}

These commands allow you to move between different branches.

\textbf{Renaming a branch:}
\begin{lstlisting}
git branch -m <new-branch-name>
\end{lstlisting}

Useful for reorganizing branch names after creation.

\textbf{Deleting a branch:}
\begin{lstlisting}
git branch -d <branch-name>
\end{lstlisting}

Removes a branch that has been merged. Use `-D' instead if you need to force-delete an unmerged branch.

\textbf{Pushing a branch to a remote repository:}
\begin{lstlisting}
git push -u origin <branch-name>
\end{lstlisting}

This command sends the branch to the remote server for collaboration.

\textbf{Fetching remote branches:}
\begin{lstlisting}
git fetch --all
\end{lstlisting}

Syncs your local repository with all available remote branches.

\textbf{Deleting a remote branch:}
\begin{lstlisting}
git push origin --delete <branch-name>
\end{lstlisting}

Used to clean up obsolete branches from the remote repository.

\textbf{Merging a branch into the current branch:}
\begin{lstlisting}
git merge <branch-name>
\end{lstlisting}

Brings the changes from another branch into the active branch.

\textbf{Checking branch status:}
\begin{lstlisting}
git status
\end{lstlisting}

Displays the current branch, uncommitted changes, and files staged for commit.

\subsection{Creating a Pull Request in Git}

A **Pull Request (PR)** allows you to propose changes to a repository and request a review before merging. Follow these steps:

\textbf{1. Create a new branch:}
\begin{lstlisting}
git checkout -b feature-branch
\end{lstlisting}
This creates a new branch for your changes.

\textbf{2. Make changes and commit:}
Edit files, then add and commit your changes:
\begin{lstlisting}
git add .
git commit -m ''Added new feature''
\end{lstlisting}

\textbf{3. Push the branch to the remote repository:}
\begin{lstlisting}
git push origin feature-branch
\end{lstlisting}

\textbf{4. Create a Pull Request on GitHub:}
1. Go to the repository on GitHub.
2. Click **''Pull requests''**, then **''New pull request''**.
3. Select `main' as the base and `feature-branch' as the compare branch.
4. Add a title and description.
5. Click **''Create pull request''**.

\textbf{5. Review and Merge:}
Once reviewed, merge the PR:\\
\begin{lstlisting}
git merge feature-branch
\end{lstlisting}

Now your changes are successfully integrated!

\subsection{Resolving Merge Conflicts in Git}

Sometimes, when merging branches, Git detects conflicts. A file with conflicts may look like this:

\begin{lstlisting}
/<<<<<<< HEAD
Your current branch changes
=======
The other branch's changes
>>>>>>> branch-name/
\end{lstlisting}

To resolve the conflict:
1. Manually edit the file to **keep the correct changes**.
2. Remove the conflict markers (`\textless\textless\textless\textless\textless\textless\textless', `=======', `\textgreater\textgreater\textgreater\textgreater\textgreater\textgreater\textgreater').
3. Stage and commit the resolved file:
\begin{lstlisting}
git add <conflicted-file>
git commit -m ''Resolved merge conflict''
\end{lstlisting}

Now the conflict is resolved, and the merge is 

\begin{longtable}{|l|p{6cm}|p{6cm}|}
    \hline
    \textbf{Command} & \textbf{Description} & \textbf{Usage Scenario} \\
    \hline
    \endhead
    git init & Initializes a new Git repository & Starting a new project \\
    \hline
    git clone \textless repo-url\textgreater & Clones an existing repository & Getting a copy of a remote project \\
    \hline
    git add \textless file\textgreater & Stages a file for commit & Preparing changes for commit \\
    \hline
    git commit -m ''message'' & Saves changes with a message & Recording changes in history \\
    \hline
    git status & Shows the current state of the repository & Checking which files are modified or staged \\
    \hline
    git log & Displays commit history & Reviewing past changes \\
    \hline
    git diff & Shows differences between commits or branches & Comparing changes before committing \\
    \hline
    git branch & Lists all branches & Checking available branches \\
    \hline
    git checkout \textless branch\textgreater & Switches to another branch & Working on a different feature \\
    \hline
    git switch \textless branch\textgreater & Alternative to checkout for switching branches & Moving between branches efficiently \\
    \hline
    git merge \textless branch\textgreater & Merges another branch into the current one & Combining changes from different branches \\
    \hline
    git pull origin \textless branch\textgreater & Fetches and merges changes from a remote repository & Updating local repository with latest changes \\
    \hline
    git push origin \textless branch\textgreater & Sends local commits to a remote repository & Sharing changes with others \\
    \hline
    git remote -v & Shows remote repository URLs & Checking remote connections \\
    \hline
    git reset --hard \textless commit\textgreater & Resets repository to a specific commit & Undoing changes completely \\
    \hline
    git stash & Temporarily saves uncommitted changes & Switching tasks without committing \\
    \hline
    git tag \textless tag-name\textgreater & Creates a tag for a specific commit & Marking important versions \\
    \hline
    git rm \textless file\textgreater & Removes a file from the repository & Deleting files from version control \\
    \hline
    git show \textless commit\textgreater & Displays details of a specific commit & Inspecting changes in a commit \\
    \hline
\end{longtable}

\subsection{Bash Commands and Their Uses}

\begin{longtable}{|l|p{6cm}|p{6cm}|}
    \hline
    \textbf{Command} & \textbf{Description} & \textbf{Usage Scenario} \\
    \hline
    \endhead
    ls & Lists files in a directory & Checking available files \\
    \hline
    cd \textless directory\textgreater & Changes the current directory & Navigating through folders \\
    \hline
    mkdir \textless directory\textgreater & Creates a new directory & Organizing files \\
    \hline
    rm -rf \textless directory\textgreater & Deletes a directory and its contents & Removing unwanted files \\
    \hline
    touch \textless file\textgreater & Creates an empty file & Initializing new files \\
    \hline
    echo ''text'' \textgreater file.txt & Writes text to a file & Creating simple text files \\
    \hline
    cat \textless file\textgreater & Displays file contents & Viewing file data \\
    \hline
    grep ``pattern'' \textless file\textgreater & Searches for a pattern in a file & Finding specific text in logs \\
    \hline
    chmod +x \textless file\textgreater & Makes a file executable & Running scripts \\
    \hline
    ps aux & Lists running processes & Checking system activity \\
    \hline
    kill \textless PID\textgreater & Terminates a process & Stopping unresponsive applications \\
    \hline
    tar -cvf archive.tar \textless directory\textgreater & Creates a compressed archive & Backing up files \\
    \hline
    scp \textless file\textgreater user\@server:/path & Securely copies files to a remote server & Transferring files over SSH \\
    \hline
    ping \textless host\textgreater & Checks network connectivity & Troubleshooting network issues \\
    \hline
    curl \textless URL\textgreater & Fetches data from a URL & Testing API responses \\
    \hline
\end{longtable}

\subsection{Setting Up SSH}

Follow these steps to configure SSH for secure access.

\textbf{1. Check if SSH is installed:}
\begin{lstlisting}
ssh -V
\end{lstlisting}

\textbf{2. Generate an SSH key:}
\begin{lstlisting}
ssh-keygen -t rsa -b 4096 -C ''your_email@example.com''
\end{lstlisting}

Press **Enter** to save it to the default location (`\verb|~/.ssh/id_rsa|`).

\textbf{3. Add the SSH key to your agent:}
\begin{lstlisting}
eval ''$(ssh-agent -s)''
ssh-add ~/.ssh/id_rsa
\end{lstlisting}

\textbf{4. Copy and add the SSH key to a remote service (e.g., GitHub):}
\begin{lstlisting}
cat ~/.ssh/id_rsa.pub
\end{lstlisting}

Copy the output and add it to your GitHub SSH settings.

\textbf{5. Test the SSH connection:}
\begin{lstlisting}
ssh -T git@github.com
\end{lstlisting}

If successful, your SSH setup is complete!

\section{Python Key tricks}

\textbf{1. Efficient Data Loading with Pandas}
\begin{minted}{python}
import pandas as pd

df = pd.read_csv(''data.csv'', low_memory=False)
df = pd.read_csv(''data.csv'', dtype={'column_name': 'float32'})
\end{minted}
Using `\verb|low_memory=False|' prevents memory fragmentation, and specifying `dtype' reduces memory usage.

\textbf{2. Vectorized Operations with NumPy}
\begin{minted}{python}
import numpy as np

X = np.random.rand(1000000)
X_squared = X ** 2  # Vectorized operation (faster than loops)
\end{minted}
Vectorization eliminates slow Python loops, improving performance.

\textbf{3. Memory-Efficient Generators for Large Datasets}
\begin{minted}{python}
def data_generator(file_path):
    with open(file_path, ''r'') as f:
        for line in f:
            yield line.strip().split('','')  # Process line-by-line

for data in data_generator(''large_data.csv''):
    print(data)
\end{minted}
Generators prevent excessive memory usage when handling large datasets.

\textbf{4. Parallel Processing with Multiprocessing}
\begin{minted}{python}
from multiprocessing import Pool

def square(n):
    return n * n

numbers = [1, 2, 3, 4, 5]
with Pool(4) as p:
    results = p.map(square, numbers)

print(results)
\end{minted}
Multiprocessing speeds up computations by **parallelizing operations**.

\textbf{5. Custom Data Preprocessing Without Sklearn}
\begin{minted}{python}
import numpy as np

def normalize(X):
    return (X - np.min(X, axis=0)) / (np.max(X, axis=0) - np.min(X, axis=0))

X = np.array([[1, 2], [3, 4], [5, 6]])
X_normalized = normalize(X)
print(X_normalized)
\end{minted}
Custom normalization avoids reliance on third-party ML libraries.

\textbf{6. Loading and Preprocessing Data Using TensorFlow}
\begin{minted}{python}
import tensorflow as tf

dataset = tf.data.TextLineDataset(''data.csv'')
dataset = dataset.map(lambda line: tf.strings.split(line, '',''))
for data in dataset.take(5):
    print(data.numpy())
\end{minted}
Using `tf.data` provides efficient streaming for large datasets.

\textbf{7. PyTorch-Based Neural Network for ML}
\begin{minted}{python}
import torch.nn as nn
import torch.optim as optim
import torch

class NeuralNet(nn.Module):
    def __init__(self):
        super(NeuralNet, self).__init__()
        self.fc1 = nn.Linear(10, 64)
        self.fc2 = nn.Linear(64, 1)

    def forward(self, x):
        x = torch.relu(self.fc1(x))
        x = torch.sigmoid(self.fc2(x))
        return x

model = NeuralNet()
optimizer = optim.Adam(model.parameters(), lr=0.001)
\end{minted}
PyTorch offers flexible ML models with **custom architecture**.

\textbf{8. Using TensorFlow for Deep Learning}
\begin{minted}{python}
model = tf.keras.Sequential([
    tf.keras.layers.Dense(64, activation='relu'),
    tf.keras.layers.Dense(1, activation='sigmoid')
])
model.compile(optimizer='adam', loss='binary_crossentropy', metrics=['accuracy'])
model.fit(X_train, y_train, epochs=10, batch_size=32)
\end{minted}
TensorFlow simplifies deep learning training without external dependencies.

\textbf{9. Making Predictions with TensorFlow and PyTorch}
\begin{minted}{python}
# TensorFlow
y_pred = model.predict(X_test)

# PyTorch
X_test_tensor = torch.tensor(X_test, dtype=torch.float32)
y_pred_pytorch = model(X_test_tensor).detach().numpy()
\end{minted}
Both TensorFlow and PyTorch provide easy inference methods.

\textbf{10. Saving and Loading Models Without Sklearn}
\begin{minted}{python}
# TensorFlow
model.save(''tf_model.h5'')
loaded_model = tf.keras.models.load_model(''tf_model.h5'')

# PyTorch
torch.save(model.state_dict(), ''pytorch_model.pth'')
model.load_state_dict(torch.load(''pytorch_model.pth''))
\end{minted}
Saving and loading models ensures reproducibility in ML workflows.

\subsection{1. Define the Problem}

Before building a Machine Learning model, it is essential to understand the objective:
\begin{itemize}
    \item Identify whether the task is \textbf{classification}, \textbf{regression}, or \textbf{clustering}.
    \item Determine the data sources required.
    \item Establish success metrics (e.g. accuracy, precision, mean squared error).
\end{itemize}

\subsection{2. Collect and Preprocess Data}

Data preparation is crucial for model accuracy. Steps include:
\begin{minted}{python}
import pandas as pd

# Load data
df = pd.read_csv(''dataset.csv'')

# Check for missing values
print(df.isnull().sum())

# Handle missing data
df.fillna(df.mean(), inplace=True)

# Convert categorical variables to numerical
df[''category''] = df[''category''].astype(''category'').cat.codes
\end{minted}

\subsection{3. Split the Data}

Dividing the data ensures reliable model evaluation:
\begin{minted}{python}
import numpy as np


# Custom train-test split without sklearn
def split_data(X, y, test_size=0.2):
    split = int(len(X) * (1 - test_size))
    return X[:split], X[split:], y[:split], y[split:]

X = df.drop(''target'', axis=1).values
y = df[''target''].values
X_train, X_test, y_train, y_test = split_data(X, y)
\end{minted}

\subsection{4. Choose the Model Type}

Select an appropriate model depending on the problem:
\begin{itemize}
    \item \textbf{Linear models}: Suitable for numerical trends.
    \item \textbf{Decision trees and random forests}: Effective for structured data.
    \item \textbf{Neural networks}: Ideal for deep learning and complex patterns.
\end{itemize}

\subsection{5. Train the Model}

The model is trained using the provided dataset:
\begin{minted}{python}
import tensorflow as tf

# Define a simple neural network model
model = tf.keras.Sequential([
    tf.keras.layers.Dense(64, activation='relu'),
    tf.keras.layers.Dense(1, activation='sigmoid')
])

# Compile and train the model
model.compile(optimizer='adam', loss='binary_crossentropy', metrics=['accuracy'])
model.fit(X_train, y_train, epochs=10, batch_size=32)
\end{minted}

\subsection{6. Evaluate the Model}

Assess model performance using various metrics:
\begin{minted}{python}
# Evaluate accuracy
loss, accuracy = model.evaluate(X_test, y_test)
print(f''Test Accuracy: {accuracy:.2f}'')
\end{minted}

\subsection{7. Tune Hyperparameters}

Optimising hyperparameters improves model efficiency:
\begin{minted}{python}
# Adjust learning rate
model.compile(optimizer=tf.keras.optimizers.Adam(learning_rate=0.001),
              loss='binary_crossentropy',
              metrics=['accuracy'])
\end{minted}

\subsection{8. Deploy the Model}

Once trained, the model can be deployed for real-world use:
\begin{minted}{python}
# Save the model
model.save(''final_model.h5'')

# Load the model for inference
loaded_model = tf.keras.models.load_model(''final_model.h5'')
\end{minted}

\subsection{9. Monitor and Improve the Model}

Continual monitoring ensures long-term success:
\begin{itemize}
    \item Track performance using automated logging.
    \item Retrain the model periodically with new data.
    \item Implement MLOps frameworks for efficient model updates.
\end{itemize}

\end{document}