\documentclass[a4paper,12pt]{article}

%----------------------------------------------------------------------------------------
%	FONT
%----------------------------------------------------------------------------------------

%----------------------------------------------------------------------------------------
%	PACKAGES
%----------------------------------------------------------------------------------------
\usepackage{url}
\usepackage{parskip} 	
\usepackage{tabularx}

%other packages for formatting
\RequirePackage{color}
\RequirePackage{graphicx}
\usepackage[usenames,dvipsnames]{xcolor}
\usepackage[scale=0.9]{geometry}
\usepackage{amsmath}
\usepackage{tikz}
\usetikzlibrary{shapes,arrows}
\usepackage{rotating}
\usetikzlibrary{shapes.geometric, arrows}
\usepackage{float}
\usepackage{algorithm}
\usepackage{algpseudocode}
\usetikzlibrary{positioning, calc}

\usepackage{float}
\usepackage{algorithm}
\usepackage{algpseudocode}
\usepackage{tikz}
\usetikzlibrary{arrows.meta, positioning}
\usepackage{setspace}
\usepackage{titlesec}
\usepackage{tikz}
\usepackage{tabularx}
\usepackage{circuitikz}
\usepackage{pgfplots}
\usepackage{amssymb}
\usepackage{amsthm}
\usepackage{minted}
\usepackage{float}  
\usepackage{booktabs}
\usepackage{xcolor}
\usepackage{longtable}
\pgfplotsset{compat=1.18}
\usepackage{setspace}
\theoremstyle{definition}
\newtheorem{definition}{Definition}
\usepackage{amsthm}
\usepackage{adjustbox}

\tikzstyle{startstop} = [rectangle, rounded corners, minimum width=3cm, minimum height=1cm,text centered, draw=black, fill=red!30]
\tikzstyle{process} = [rectangle, minimum width=3cm, minimum height=1cm, text centered, draw=black, fill=blue!30]
\tikzstyle{decision} = [diamond, minimum width=3cm, minimum height=1cm, text centered, draw=black, fill=green!30]
\tikzstyle{arrow} = [thick,->,>=stealth]

\definecolor{codegreen}{rgb}{0,0.6,0}
\definecolor{codegray}{rgb}{0.5,0.5,0.5}
\definecolor{codepurple}{rgb}{0.58,0,0.82}
\definecolor{backcolour}{rgb}{0.95,0.95,0.92}

\usepackage{listings}

\lstdefinestyle{mystyle}{
    backgroundcolor=\color{backcolour},   
    commentstyle=\color{codegreen},
    keywordstyle=\color{magenta},
    numberstyle=\tiny\color{codegray},
    stringstyle=\color{codepurple},
    basicstyle=\ttfamily\footnotesize,
    breakatwhitespace=false,         
    breaklines=true,                 
    captionpos=b,                    
    keepspaces=true,                 
    numbers=left,                    
    numbersep=5pt,                  
    showspaces=false,                
    showstringspaces=false,
    showtabs=false,                  
    tabsize=2
}

\lstset{style=mystyle}

%tabularx environment
\usepackage{tabularx}

%for lists within experience section
\usepackage{enumitem}

% centered version of 'X' col. type
\newcolumntype{C}{>{\centering\arraybackslash}X} 

%to prevent spillover of tabular into next pages
\usepackage{supertabular}
\usepackage{tabularx}
\newlength{\fullcollw}
\setlength{\fullcollw}{0.47\textwidth}

%custom \section
\usepackage{titlesec}				
\usepackage{multicol}
\usepackage{multirow}

%CV Sections inspired by: 
%http://stefano.italians.nl/archives/26
\titleformat{\section}{\large\scshape\raggedright}{}{0em}{}[\titlerule]
\titlespacing{\section}{0pt}{10pt}{10pt}

%for publications
\usepackage[style=authoryear,sorting=ynt, maxbibnames=2]{biblatex}

%Setup hyperref package, and colours for links
\usepackage[unicode, draft=false]{hyperref}
\definecolor{linkcolour}{rgb}{0,0.2,0.6}
\hypersetup{colorlinks,breaklinks,urlcolor=linkcolour,linkcolor=linkcolour}
\addbibresource{citations.bib}
\setlength\bibitemsep{1em}

%for social icons
\usepackage{fontawesome5}


% Remove the date
\date{}

% Title format

\begin{document}

\onehalfspacing
% Title Page
\begin{titlepage}
    \centering
    \vspace*{2cm}
    \Huge{\textbf{Python Shortcuts for internship}} \\[1.5cm]
    Sahas Talasila \\[1cm]
    \vfill
\end{titlepage}

% Table of Contents
\tableofcontents
\newpage

\section{Introduction}

\begin{minted}[bgcolor=gray!10]{python}
# variables.py - Understanding Variables and Data Types

# Variables are used to store values in Python.
# You don’t need to declare types explicitly; Python infers them.

# 1. Integer variable
age = 25  # A whole number
print("Age:", age, "| Type:", type(age))  # Outputs: Age: 25 | Type: <class 'int'>

# 2. Float variable
height = 5.9  # A decimal (floating-point) number
print("Height:", height, "| Type:", type(height))  
# Outputs: Height: 5.9 | Type: <class 'float'>

# 3. String variable
name = "Sahas"  # Text (a sequence of characters)
print("Name:", name, "| Type:", type(name))  
# Outputs: Name: Sahas | Type: <class 'str'>

# 4. Boolean variable
is_learning = True  # Boolean value (True or False)
print("Learning:", is_learning, "| Type:", type(is_learning))  
# Outputs: Learning: True | Type: <class 'bool'>

# 5. Lists - Ordered collections of items
numbers = [1, 2, 3, 4, 5]  # A list of integers
print("Numbers:", numbers, "| Type:", type(numbers))  
# Outputs: Numbers: [1, 2, 3, 4, 5] | Type: <class 'list'>

# 6. Dictionaries - Key-value pairs for storing structured data
person = {"name": "Sahas", "age": 25, "height": 5.9}
print("Person:", person, "| Type:", type(person))  
#Outputs: Person: {'name': 'Sahas', 'age': 25, 'height': 5.9} | Type: <class 'dict'>

# Type conversion
age_as_string = str(age)  # Convert integer to string
height_as_int = int(height)  # Convert float to integer (loss of precision)
print("Converted age:", age_as_string, "| Type:", type(age_as_string))  
# Outputs: Converted age: 25 | Type: <class 'str'>
print("Converted height:", height_as_int, "| Type:", type(height_as_int))  
# Outputs: Converted height: 5 | Type: <class 'int'>
\end{minted}

\subsection{Control Flow}

\begin{minted}[bgcolor=gray!20]{python}
# control_flow.py - Understanding Control Flow in Python

# Control flow structures determine the logic of how a program executes.

### Conditional Statements (if, elif, else)
age = 20

if age < 18:
    print("You are a minor.")
elif 18 <= age < 65:
    print("You are an adult.")
else:
    print("You are a senior citizen.")

# Explanation:
# - The `if` block executes only if the condition `age < 18` is True.
# - The `elif` block executes if `18 <= age < 65` is True.
# - If none of the above conditions are met, the `else` block runs.

### Loops (for and while)
# `for` loop: Iterating over a sequence (list, tuple, range)
numbers = [10, 20, 30, 40, 50]

print("Iterating using a for loop:")
for num in numbers:
    print(num)  # Prints each number in the list

# `while` loop: Repeats execution while a condition remains True
count = 0
print("Iterating using a while loop:")
while count < 5:
    print("Count is:", count)
    count += 1  # Increment count (prevents infinite loop)

### Exception Handling
# Prevents the program from crashing due to runtime errors
try:
    x = int(input("Enter a number: "))  # User inputs a value
    result = 10 / x  # May cause division by zero
    print("Result:", result)
except ZeroDivisionError:
    print("Error: Cannot divide by zero!")
except ValueError:
    print("Error: Invalid input! Please enter a valid number.")
finally:
    print("Execution complete.")

# Explanation:
# - `try`: Code that may cause an error.
# - `except ZeroDivisionError`: Handles cases where division by zero occurs.
# -'except ValueError': Catches invalid input (e.g., user enters non-numeric values).
# - `finally`: Executes regardless of errors.
\end{minted}

\subsection{Functions and Recursion}

\begin{minted}[bgcolor=gray!20]{python}
    # functions.py - Understanding Functions and Recursion in Python

# Functions allow reusable blocks of code that can be executed multiple times.

### Defining and Calling Functions
def greet(name):
    """Function to greet a user by name."""
    print(f"Hello, {name}! Welcome!")

# Calling the function
greet("Sahas")  # Outputs: Hello, Sahas! Welcome!


### Arguments and Return Values
def add_numbers(a, b):
    """Function that returns the sum of two numbers."""
    return a + b  # The `return` statement sends back a result

# Storing function output in a variable
result = add_numbers(5, 7)
print("Sum:", result)  # Outputs: Sum: 12


### Default Arguments
def power(base, exponent=2):
    """Function with a default exponent of 2 (square)."""
    return base ** exponent

print("Default exponent (square):", power(3))   # Outputs: exponent (square): 9
print("Custom exponent:", power(2, 3))         # Outputs: Custom exponent: 8


### Recursion - When a function calls itself
def factorial(n):
    """Computes factorial using recursion (n! = n × (n-1) × (n-2)...×1)."""
    if n == 0 or n == 1:  # Base case: Factorial of 0 or 1 is always 1
        return 1
    return n * factorial(n - 1)  # Recursive case: n multiplied by factorial(n-1)

print("Factorial of 5:", factorial(5))  # Outputs: Factorial of 5: 120

# Explanation:
# - Base case prevents infinite recursion.
# - Each call reduces `n` until it reaches 1.
# - The function builds the result step-by-step as it returns from recursion.
\end{minted}

\subsection{OOP Python edition}

\begin{minted}[bgcolor=gray!20]{python}
# oop.py - Deep Dive into Object-Oriented Programming (OOP)

# A class is a blueprint for creating objects. An object is an instance of a class.

### 1. Defining a Class and Creating Objects
class Person:
    """A class that represents a person."""

    def __init__(self, name, age):
        """Constructor method (__init__): Initializes object attributes."""
        self.name = name  # Instance attribute
        self.age = age    # Instance attribute

    def introduce(self):
        """Instance method: Uses attributes of the object."""
        return f"Hello, my name is {self.name} and I am {self.age} years old."

# Creating instances (objects)
person1 = Person("Sahas", 25)
person2 = Person("Alice", 30)

print(person1.introduce())  # Outputs: Hello, my name is Sahas and I am 25 years old.
print(person2.introduce())  # Outputs: Hello, my name is Alice and I am 30 years old.


### 2. Encapsulation (Restricting Direct Access to Attributes)
class BankAccount:
    """Encapsulated bank account class (uses private attributes)."""

    def __init__(self, owner, balance):
        self.owner = owner
        self.__balance = balance  # Private attribute (cannot be accessed directly)

    def deposit(self, amount):
        """Deposits money into the account."""
        if amount > 0:
            self.__balance += amount
            return f"Deposited £{amount}. New balance: £{self.__balance}"
        return "Invalid deposit amount."

    def get_balance(self):
        """Accesses the private attribute."""
        return f"Balance for {self.owner}: £{self.__balance}"

# Creating an account
account = BankAccount("Sahas", 1000)

# Accessing balance indirectly
print(account.get_balance())  # Outputs: Balance for Sahas: £1000

# Trying to access private attribute directly (fails)
# print(account.__balance)  # This would raise an AttributeError

### 3. Inheritance (Extending a Class)
# A subclass inherits attributes and methods from a parent class.

class Employee(Person):  # Employee class inherits from Person
    """Employee class extending Person."""

    def __init__(self, name, age, job_title, salary):
        """Extend constructor: Call parent class constructor using super()"""
        super().__init__(name, age)
        self.job_title = job_title
        self.salary = salary

    def work(self):
        """Instance method specific to Employee."""
        return f"{self.name} works as a {self.job_title} and 
        earns £{self.salary} annually."

# Creating an Employee object
employee1 = Employee("Sahas", 25, "Software Engineer", 50000)
print(employee1.introduce())  
# Outputs inherited method: Hello, my name is Sahas and I am 25 years old.
print(employee1.work())  
# Outputs subclass method: Sahas works as a Software Engineer and earns £50000 annually.

### 4. Polymorphism (Using Same Methods in Different Ways)
# Polymorphism allows different classes to use the same method name.

class Dog:
    """Dog class with speak method."""
    def speak(self):
        return "Woof!"

class Cat:
    """Cat class with speak method."""
    def speak(self):
        return "Meow!"

# Using polymorphism
animals = [Dog(), Cat()]
for animal in animals:
    print(animal.speak())  # Outputs: Woof! Meow!

---

### 5. Special Methods (`__init__`, `__str__`, `__repr__`)
# Python provides special methods to customize object behavior.

class Book:
    """Book class showcasing special methods."""

    def __init__(self, title, author, pages):
        """Initializer method."""
        self.title = title
        self.author = author
        self.pages = pages

    def __str__(self):
        """Readable representation of object (used when printing)."""
        return f"'{self.title}' by {self.author}, {self.pages} pages."

    def __repr__(self):
        """Technical representation (used in debugging)."""
        return f"Book(title={self.title}, author={self.author}, pages={self.pages})"

# Creating a book object
book1 = Book("Python Mastery", "Sahas", 350)
print(book1)  # Outputs: 'Python Mastery' by Sahas, 350 pages.
print(repr(book1))  # Outputs: Book(title=Python Mastery, author=Sahas, pages=350)

---

### 6. Private & Protected Variables
# Private (`__var`) and protected (`_var`) attributes control access levels.

class SecureData:
    """Demonstrates private and protected variables."""

    def __init__(self):
        self._protected_var = "This is protected"  # Accessible in subclasses
        self.__private_var = "This is private"  # Not directly accessible

    def access_private(self):
        """Accessing private attribute via method."""
        return self.__private_var

# Creating instance
data = SecureData()

print(data._protected_var)  # Outputs: This is protected
# print(data.__private_var)  # This would raise an error (private variable)

print(data.access_private())  # Outputs: This is private (accessed via method)
\end{minted}

\subsection{Extra Python OOP}

\begin{minted}[bgcolor=gray!20]{python}
# advanced_oop.py - Comprehensive Guide to Object-Oriented Programming (OOP)

# -------------------
# 1. CLASS OPERATIONS
# -------------------
class Vehicle:
    """Base class representing a generic vehicle."""
    
    # Class attribute (shared across all instances)
    vehicle_count = 0
    
    def __init__(self, brand, model, year):
        """Constructor method (__init__) - Initializes instance attributes."""
        self.brand = brand  # Instance attribute
        self.model = model
        self.year = year
        Vehicle.vehicle_count += 1  # Incrementing class attribute

    def display_info(self):
        """Instance method - Provides details about the vehicle."""
        return f"{self.year} {self.brand} {self.model}"

# Creating instances
car1 = Vehicle("Toyota", "Corolla", 2020)
car2 = Vehicle("Ford", "Focus", 2022)

print(car1.display_info())  # Outputs: 2020 Toyota Corolla
print(car2.display_info())  # Outputs: 2022 Ford Focus
print("Total Vehicles:", Vehicle.vehicle_count)  # Outputs: Total Vehicles: 2

# -------------------
# 2. ENCAPSULATION & ACCESS MODIFIERS
# -------------------

class BankAccount:
    """Encapsulation Example - Restricting direct access to attributes."""
    
    def __init__(self, owner, balance):
        self.owner = owner
        self._protected_balance = balance 
        # Protected variable (can be accessed by subclasses)
        self.__private_balance = balance  
        # Private variable (cannot be accessed directly)

    def deposit(self, amount):
        """Deposits money."""
        if amount > 0:
            self.__private_balance += amount
            return f"Deposited £{amount}. New balance: £{self.__private_balance}"
        return "Invalid deposit amount."

    def get_balance(self):
        """Access private attribute via method."""
        return f"Balance for {self.owner}: £{self.__private_balance}"

# Creating an account
account = BankAccount("Sahas", 1000)
print(account.get_balance())  # Outputs: Balance for Sahas: £1000
# print(account.__private_balance)  # This would raise an AttributeError

# -------------------
# 3. INHERITANCE & METHOD OVERRIDING
# -------------------
class Employee:
    """Base class for employees."""
    def __init__(self, name, salary):
        self.name = name
        self.salary = salary

    def get_info(self):
        return f"{self.name} earns £{self.salary} per year."

class Developer(Employee):  # Developer inherits from Employee
    def __init__(self, name, salary, programming_language):
        super().__init__(name, salary)  # Call parent class constructor
        self.programming_language = programming_language

    def get_info(self):  # Method overriding
        return f"{self.name} writes {self.programming_language} code and 
        earns £{self.salary}."

dev1 = Developer("Sahas", 60000, "Python")
print(dev1.get_info())  # Outputs: Sahas writes Python code and earns £60000.

# -------------------
# 4. ABSTRACT CLASSES & INTERFACES
# -------------------
from abc import ABC, abstractmethod  # Import abstract class module

class Shape(ABC):  # Abstract base class
    """Abstract class representing geometric shapes."""
    
    @abstractmethod
    def area(self):
        """Abstract method - Must be implemented by subclasses."""
        pass

class Circle(Shape):
    def __init__(self, radius):
        self.radius = radius

    def area(self):
        """Implements required abstract method."""
        return 3.14 * self.radius ** 2

circle1 = Circle(5)
print("Circle Area:", circle1.area())  # Outputs: Circle Area: 78.5

# -------------------
# 5. MULTIPLE INHERITANCE
# -------------------
class A:
    def method_a(self):
        return "Method A"

class B:
    def method_b(self):
        return "Method B"

class C(A, B):  # Multiple inheritance
    def method_c(self):
        return "Method C"

obj = C()
print(obj.method_a())  # Outputs: Method A
print(obj.method_b())  # Outputs: Method B
print(obj.method_c())  # Outputs: Method C


# -------------------
# 6. OPERATOR OVERLOADING
# -------------------
class Vector:
    """Operator overloading example - Adding two vectors using + operator."""
    
    def __init__(self, x, y):
        self.x = x
        self.y = y

    def __add__(self, other):  # Overloading the + operator
        return Vector(self.x + other.x, self.y + other.y)

    def __str__(self):
        return f"Vector({self.x}, {self.y})"

v1 = Vector(2, 3)
v2 = Vector(4, 5)
result = v1 + v2  # Uses overloaded + operator
print(result)  # Outputs: Vector(6, 8)


# -------------------
# 7. METAPROGRAMMING (CLASS ATTRIBUTES & DECORATORS)
# -------------------
def log_method(func):
    """Custom decorator to log method calls."""
    def wrapper(*args, **kwargs):
        print(f"Calling {func.__name__}...")
        return func(*args, **kwargs)
    return wrapper

class Logger:
    """Example of using decorators in a class."""
    
    @log_method  # Applying decorator
    def action(self):
        return "Action executed!"

logger = Logger()
print(logger.action())  # Outputs: Calling action... Action executed!


# -------------------
# 8. DESIGN PATTERNS - SINGLETON
# -------------------
class Singleton:
    """Singleton pattern ensures only one instance is created."""
    
    _instance = None

    def __new__(cls):
        if cls._instance is None:
            cls._instance = super().__new__(cls)
        return cls._instance

s1 = Singleton()
s2 = Singleton()
print(s1 is s2)  # Outputs: True (both references point to the same instance)
\end{minted}

\subsection{File Formats and Conversion}

\begin{minted}[bgcolor=gray!20]{python}
# full_file_operations.py - Comprehensive File Handling Operations in Python

import os
import json
import csv
import shutil  # For copying/moving files
import zipfile  # For ZIP compression
import tarfile  # For TAR compression

# -------------------
# 1. TEXT FILE OPERATIONS
# -------------------

file_path = "example.txt"

# Writing (overwrite mode 'w')
with open(file_path, "w") as file:
    file.write("Hello, Sahas!\nWelcome to Python file handling.\n")

# Appending ('a' mode)
with open(file_path, "a") as file:
    file.write("Appending new content.\n")

# Reading ('r' mode)
with open(file_path, "r") as file:
    content = file.read()
    print("Text File Content:\n", content)

# Reading line-by-line
with open(file_path, "r") as file:
    for line in file:
        print("Line:", line.strip())

# -------------------
# 2. JSON FILE OPERATIONS
# -------------------

json_data = {"name": "Sahas", "age": 25, "languages": ["Python", "C", "Lua"]}

# Writing JSON ('w' mode)
with open("data.json", "w") as json_file:
    json.dump(json_data, json_file, indent=4)

# Reading JSON ('r' mode)
with open("data.json", "r") as json_file:
    loaded_json = json.load(json_file)
    print("Loaded JSON:", loaded_json)

# -------------------
# 3. CSV FILE OPERATIONS
# -------------------

csv_file = "data.csv"

# Writing CSV ('w' mode)
with open(csv_file, "w", newline="") as csv_file:
    writer = csv.writer(csv_file)
    writer.writerow(["Name", "Age", "Language"])
    writer.writerow(["Sahas", "25", "Python"])
    writer.writerow(["Alice", "30", "C++"])

# Reading CSV ('r' mode)
with open(csv_file, "r") as csv_file:
    reader = csv.reader(csv_file)
    for row in reader:
        print("CSV Row:", row)

# -------------------
# 4. BINARY FILE OPERATIONS
# -------------------

binary_file = "binary_file.bin"
binary_data = b"\x89PNG\r\n\x1a\n\x00\x00\x00IHDR"

# Writing Binary ('wb' mode)
with open(binary_file, "wb") as bin_file:
    bin_file.write(binary_data)

# Reading Binary ('rb' mode)
with open(binary_file, "rb") as bin_file:
    loaded_bin = bin_file.read()
    print("Loaded Binary Data:", loaded_bin)

# -------------------
# 5. FILE PATH OPERATIONS
# -------------------

print("Current Directory:", os.getcwd())
print("Does 'example.txt' exist?", os.path.exists(file_path))
print("Absolute Path:", os.path.abspath(file_path))

# -------------------
# 6. FILE MANAGEMENT (COPYING/MOVING/DELETING)
# -------------------

shutil.copy(file_path, "copy_example.txt")  # Copy file
shutil.move("copy_example.txt", "moved_example.txt")  # Move file
os.remove("moved_example.txt")  # Delete file

# -------------------
# 7. FILE COMPRESSION (ZIP & TAR)
# -------------------

# Create a ZIP file
with zipfile.ZipFile("compressed.zip", "w") as zipf:
    zipf.write(file_path)  # Add file to ZIP

# Extract ZIP file
with zipfile.ZipFile("compressed.zip", "r") as zipf:
    zipf.extractall("extracted_files")  # Extract to folder

# Create a TAR file
with tarfile.open("compressed.tar.gz", "w:gz") as tarf:
    tarf.add(file_path)  # Add file to TAR

# Extract TAR file
with tarfile.open("compressed.tar.gz", "r:gz") as tarf:
    tarf.extractall("extracted_tar_files")

# -------------------
# 8. ERROR HANDLING IN FILE OPERATIONS
# -------------------

try:
    with open("non_existent.txt", "r") as file:
        content = file.read()
except FileNotFoundError:
    print("Error: File does not exist!")
except IOError:
    print("Error: Issue with reading the file.")
\end{minted}

\section{Machine Learning Basics}

\subsection{NumPy}

\begin{minted}[bgcolor=gray!20]{python}
# full_numpy_operations.py - Comprehensive NumPy Operations in Python

import numpy as np

# -------------------
# 1. ARRAY CREATION & INITIALIZATION
# -------------------

# Creating NumPy arrays from lists
arr1 = np.array([1, 2, 3, 4])
arr2 = np.array([[1, 2, 3], [4, 5, 6]])  # 2D array

# Special arrays
zeros = np.zeros((3, 3))  # 3x3 zero matrix
ones = np.ones((2, 2))  # 2x2 ones matrix
identity_matrix = np.eye(4)  # 4x4 identity matrix
random_array = np.random.rand(3, 3)  # Random values in range [0,1]
range_array = np.arange(1, 10, 2)  # [1, 3, 5, 7, 9]
linspace_array = np.linspace(0, 1, 5)  # 5 evenly spaced values between 0 and 1

print("Array from list:\n", arr1)
print("Identity Matrix:\n", identity_matrix)
print("Linspace Array:\n", linspace_array)

# -------------------
# 2. ARRAY SHAPE, SIZE & DATA TYPES
# -------------------

print("Shape of arr2:", arr2.shape)  # (2, 3)
print("Size of arr2:", arr2.size)  # Total number of elements
print("Data type of arr1:", arr1.dtype)  # Data type of elements

# Changing data type
arr_float = arr1.astype(float)  # Convert to float type

# -------------------
# 3. INDEXING, SLICING & ADVANCED SELECTION
# -------------------

print("First element:", arr1[0])  # Access single element
print("First row of arr2:", arr2[0])  # Access row
print("Element at row 1, col 2:", arr2[1, 2])  # Access specific element

# Slicing
print("First two elements:", arr1[:2])  # First two values
print("First row, first two columns:\n", arr2[:1, :2])  # Partial matrix slice

# Advanced Indexing
bool_mask = arr1 > 2  # Boolean mask selection
filtered_values = arr1[bool_mask]  # Extract elements matching condition
print("Filtered Values:", filtered_values)

# -------------------
# 4. MATRIX OPERATIONS
# -------------------

matrix1 = np.array([[1, 2], [3, 4]])
matrix2 = np.array([[5, 6], [7, 8]])

# Basic operations
print("Matrix Addition:\n", matrix1 + matrix2)
print("Element-wise Multiplication:\n", matrix1 * matrix2)

# Dot product
dot_product = np.dot(matrix1, matrix2)
print("Dot Product:\n", dot_product)

# -------------------
# 5. LINEAR ALGEBRA FUNCTIONS
# -------------------

print("Transpose:\n", matrix1.T)
print("Inverse:\n", np.linalg.inv(matrix1))
print("Determinant:", np.linalg.det(matrix1))
print("Eigenvalues:", np.linalg.eigvals(matrix1))  # Eigenvalues of matrix
print("QR Decomposition:\n", np.linalg.qr(matrix1))  # QR factorization

# -------------------
# 6. STATISTICAL FUNCTIONS
# -------------------

stats_array = np.array([[1, 2, 3], [4, 5, 6]])

print("Mean:", np.mean(stats_array))
print("Median:", np.median(stats_array))
print("Variance:", np.var(stats_array))
print("Standard Deviation:", np.std(stats_array))
print("Min:", np.min(stats_array))
print("Max:", np.max(stats_array))

# -------------------
# 7. SORTING, SEARCHING & FILTERING
# -------------------

sorted_arr = np.sort(arr2, axis=1)  # Sort rows
print("Sorted Array:\n", sorted_arr)

unique_values = np.unique(arr2)  # Unique elements
print("Unique Values:", unique_values)

# -------------------
# 8. CONCATENATION, STACKING & BROADCASTING
# -------------------

arr_a = np.array([[1, 2], [3, 4]])
arr_b = np.array([[5, 6], [7, 8]])

concat_horiz = np.hstack((arr_a, arr_b))  # Horizontal stack
concat_vert = np.vstack((arr_a, arr_b))  # Vertical stack

print("Horizontal Concatenation:\n", concat_horiz)
print("Vertical Concatenation:\n", concat_vert)

vector = np.array([1, 2])
broadcasted_matrix = matrix1 + vector  # Adds vector to every row
print("Broadcasted Addition:\n", broadcasted_matrix)

# -------------------
# 9. ADVANCED NUMPY OPERATIONS
# -------------------

# Reshaping arrays
reshaped = arr1.reshape((2, 2))
print("Reshaped Array:\n", reshaped)

# Flattening a matrix
flattened = matrix1.flatten()
print("Flattened Matrix:", flattened)

# -------------------
# 10. FOURIER TRANSFORM & POLYNOMIAL FITTING
# -------------------

signal = np.array([0, 1, 0, -1])
fft_result = np.fft.fft(signal)  # Fast Fourier Transform
print("FFT Result:", fft_result)

# Polynomial fitting
x = np.array([0, 1, 2, 3])
y = np.array([1, 3, 7, 13])
coefficients = np.polyfit(x, y, 2)  # Fit quadratic polynomial
print("Polynomial Coefficients:", coefficients)

# -------------------
# 11. HISTOGRAM ANALYSIS
# -------------------

data_samples = np.random.randn(1000)  # Generate random data
histogram, bins = np.histogram(data_samples, bins=10)
print("Histogram Bins:", bins)
print("Histogram Counts:", histogram)

# -------------------
# 12. MEMORY OPTIMIZATION & PERFORMANCE
# -------------------

arr_large = np.random.rand(1000000)  # Large dataset
print("Memory size (bytes):", arr_large.nbytes)  # Checking memory usage
\end{minted}

\subsection{Matplotlib and Seaborn usage}

\begin{minted}[bgcolor=gray!20]{python}
# visualization.py - Comprehensive Data Visualization with Matplotlib & Seaborn

import numpy as np
import pandas as pd
import matplotlib.pyplot as plt
import seaborn as sns

# Generate sample data
np.random.seed(42)
data = np.random.randn(1000)  # Normal distribution

# Create a DataFrame for Seaborn
df = pd.DataFrame({
    "Category": np.random.choice(["A", "B", "C"], size=100),
    "Values": np.random.randint(1, 100, size=100)
})

# -------------------
# 1. BASIC MATPLOTLIB PLOTS
# -------------------

# Line Plot
x = np.linspace(0, 10, 100)
y = np.sin(x)
plt.figure(figsize=(8, 4))
plt.plot(x, y, label="Sine Wave", color="blue", linestyle="--")
plt.title("Line Plot Example")
plt.xlabel("X-axis")
plt.ylabel("Y-axis")
plt.legend()
plt.grid()
plt.show()

# Scatter Plot
plt.figure(figsize=(6, 4))
plt.scatter(np.random.rand(50), np.random.rand(50), color="red", marker="o")
plt.title("Scatter Plot Example")
plt.xlabel("X-axis")
plt.ylabel("Y-axis")
plt.show()

# -------------------
# 2. HISTOGRAM & PIE CHART
# -------------------

# Histogram
plt.figure(figsize=(6, 4))
plt.hist(data, bins=20, color="purple", edgecolor="black", alpha=0.7)
plt.title("Histogram Example")
plt.xlabel("Values")
plt.ylabel("Frequency")
plt.show()

# Pie Chart
sizes = [40, 30, 20, 10]
labels = ["A", "B", "C", "D"]
plt.figure(figsize=(5, 5))
plt.pie(sizes, labels=labels, autopct="%.1f%%", startangle=140)
plt.title("Pie Chart Example")
plt.show()

# -------------------
# 3. MULTIPLE SUBPLOTS
# -------------------

fig, axes = plt.subplots(2, 2, figsize=(10, 8))

# Line plot
axes[0, 0].plot(x, y, color="blue")
axes[0, 0].set_title("Line Plot")

# Scatter plot
axes[0, 1].scatter(np.random.rand(50), np.random.rand(50), color="red")
axes[0, 1].set_title("Scatter Plot")

# Histogram
axes[1, 0].hist(data, bins=20, color="green", alpha=0.7)
axes[1, 0].set_title("Histogram")

# Pie chart
axes[1, 1].pie(sizes, labels=labels, autopct="%.1f%%")
axes[1, 1].set_title("Pie Chart")

plt.tight_layout()
plt.show()

# -------------------
# 4. SEABORN VISUALIZATION TECHNIQUES
# -------------------

# Boxplot
plt.figure(figsize=(6, 4))
sns.boxplot(x=df["Category"], y=df["Values"], palette="Set2")
plt.title("Seaborn Boxplot Example")
plt.show()

# Violin Plot
plt.figure(figsize=(6, 4))
sns.violinplot(x=df["Category"], y=df["Values"], palette="coolwarm")
plt.title("Seaborn Violin Plot Example")
plt.show()

# Strip Plot (Scatter on Categories)
plt.figure(figsize=(6, 4))
sns.stripplot(x=df["Category"], y=df["Values"], jitter=True, palette="Set3")
plt.title("Seaborn Strip Plot Example")
plt.show()

# -------------------
# 5. REGRESSION & DISTRIBUTION PLOTS
# -------------------

# Regression Plot
sns.lmplot(x="Values", y="Values", hue="Category", data=df, height=5)
plt.title("Seaborn Regression Plot")
plt.show()

# Distribution Plot (Histogram + KDE)
plt.figure(figsize=(6, 4))
sns.histplot(data, kde=True, bins=30, color="darkred")
plt.title("Seaborn Distribution Plot")
plt.show()

# KDE Plot
plt.figure(figsize=(6, 4))
sns.kdeplot(data, shade=True, color="darkblue")
plt.title("Seaborn KDE Density Plot")
plt.show()

# -------------------
# 6. PAIR PLOTS & HEATMAPS
# -------------------

# Pair Plot
sns.pairplot(df, hue="Category", palette="husl")
plt.title("Seaborn Pairplot Example")
plt.show()

# Heatmap (Correlation Matrix)
corr_matrix = df.corr()
plt.figure(figsize=(6, 4))
sns.heatmap(corr_matrix, annot=True, cmap="coolwarm")
plt.title("Seaborn Heatmap Example")
plt.show()
\end{minted}

Linear regression models the relationship between an independent variable \( x \) and a dependent variable \( y \):

\[
y = \beta_0 + \beta_1 x + \epsilon
\]

where:
\begin{itemize}
    \item \( \beta_0 \) (intercept) controls the baseline prediction,
    \item \( \beta_1 \) (slope) determines the influence of \( x \) on \( y \),
    \item \( \epsilon \) represents the error (residual).
\end{itemize}

To estimate \( \beta_0 \) and \( \beta_1 \), we minimize the **sum of squared errors**, defined as:

\[
J(\beta_0, \beta_1) = \sum_{i=1}^{n} (y_i - (\beta_0 + \beta_1 x_i))^2
\]

Solving for partial derivatives:

\[
\beta_1 = \frac{\sum_{i=1}^{n} (x_i - \bar{x})(y_i - \bar{y})}{\sum_{i=1}^{n} (x_i - \bar{x})^2}
\]

\[
\beta_0 = \bar{y} - \beta_1 \bar{x}
\]

where \( \bar{x} \) and \( \bar{y} \) are the **means** of \( x \) and \( y \).

---

\subsection{Error Metrics: RMSE, SSE, R²}

\subsubsection{Root Mean Squared Error (RMSE)}

RMSE measures the standard deviation of residuals:

\[
RMSE = \sqrt{\frac{1}{n} \sum_{i=1}^{n} (y_i - \hat{y}_i)^2}
\]

where \( \hat{y}_i \) is the predicted value.

\subsubsection{Sum of Squared Errors (SSE)}

SSE quantifies total residual variation:

\[
SSE = \sum_{i=1}^{n} (y_i - \hat{y}_i)^2
\]

\subsubsection{Coefficient of Determination (R²)}

\( R^2 \) measures how well regression explains data variance:

\[
R^2 = 1 - \frac{\sum_{i=1}^{n} (y_i - \hat{y}_i)^2}{\sum_{i=1}^{n} (y_i - \bar{y})^2}
\]

where the denominator represents **total variance**.

---

\subsection{Gradient Descent Optimization}

Instead of solving equations directly, we iteratively optimize parameters using **Gradient Descent**:

\[
\beta_1^{new} = \beta_1^{old} - \alpha \frac{\partial J}{\partial \beta_1}
\]

\[
\beta_0^{new} = \beta_0^{old} - \alpha \frac{\partial J}{\partial \beta_0}
\]

where:
\begin{itemize}
    \item \( \alpha \) is the learning rate,
    \item \( \frac{\partial J}{\partial \beta} \) are gradients computed from data.
\end{itemize}

---

\subsection{Python Implementation}

Below is the Python implementation using NumPy, formatted using `minted` with a gray background.

\begin{minted}[bgcolor=gray!20]{python}
import numpy as np

# Sample data
X = np.array([1, 2, 3, 4, 5])
Y = np.array([2, 4, 5, 4, 5])

# Compute means
X_mean = np.mean(X)
Y_mean = np.mean(Y)

# Compute slope (beta_1)
numerator = np.sum((X - X_mean) * (Y - Y_mean))
denominator = np.sum((X - X_mean) ** 2)
beta_1 = numerator / denominator

# Compute intercept (beta_0)
beta_0 = Y_mean - beta_1 * X_mean

# Compute Predictions
Y_pred = beta_0 + beta_1 * X

# Compute RMSE
rmse = np.sqrt(np.mean((Y - Y_pred) ** 2))

# Compute SSE
sse = np.sum((Y - Y_pred) ** 2)

# Compute R² Score
sst = np.sum((Y - Y_mean) ** 2)
r_squared = 1 - (sse / sst)

# Print results
print(f"Linear Regression Equation: y = {beta_0:.2f} + {beta_1:.2f}x")
print(f"RMSE: {rmse:.4f}")
print(f"SSE: {sse:.4f}")
print(f"R² Score: {r_squared:.4f}")

# Gradient Descent Optimization
alpha = 0.01  # Learning rate
beta_0_gd, beta_1_gd = 0, 0  # Initial parameters

for epoch in range(1000):
    Y_pred_gd = beta_0_gd + beta_1_gd * X
    error = Y_pred_gd - Y
    
    # Compute gradients
    grad_beta_0 = np.mean(error)
    grad_beta_1 = np.mean(error * X)
    
    # Update parameters
    beta_0_gd -= alpha * grad_beta_0
    beta_1_gd -= alpha * grad_beta_1

print(f"Optimized Parameters using Gradient Descent: beta_0 = 
{beta_0_gd:.2f}, beta_1 = {beta_1_gd:.2f}")
\end{minted}

Gradient Descent is an optimization algorithm used to minimize a function by iteratively moving in the direction of the negative gradient.

For a given function \( J(\theta) \), the update rule for **Gradient Descent** is:

\[
\theta^{(t+1)} = \theta^{(t)} - \alpha \frac{\partial J}{\partial \theta}
\]

where:
\begin{itemize}
    \item \( \theta \) represents the parameters to be optimized,
    \item \( \alpha \) is the \textbf{learning rate},
    \item \( \frac{\partial J}{\partial \theta} \) is the \textbf{gradient}.
\end{itemize}

---

\section{Types of Gradient Descent}

Gradient Descent can be implemented in different ways depending on how we update the parameters.

\subsection{Batch Gradient Descent}
Batch Gradient Descent updates the parameters **using the entire dataset** at each iteration:

\[
\theta^{(t+1)} = \theta^{(t)} - \alpha \nabla J(\theta)
\]

**Pros**:
- More stable updates as gradients are computed on the full dataset.
- Converges smoothly to the optimal solution.

**Cons**:
- Computationally expensive for large datasets.
- Slower compared to other methods.

\subsection{Stochastic Gradient Descent (SGD)}
Stochastic Gradient Descent updates the parameters **using a single sample** at a time:

\[
\theta^{(t+1)} = \theta^{(t)} - \alpha \nabla J(\theta^{(i)})
\]

**Pros**:
- Faster updates since each iteration processes only one example.
- Can escape local minima due to randomness.

**Cons**:
- High variance in updates, making convergence noisier.
- Requires tuning of learning rate carefully.

\subsection{Mini-Batch Gradient Descent}
Mini-Batch Gradient Descent finds a middle ground, updating parameters **using a subset (mini-batch) of data**:

\[
\theta^{(t+1)} = \theta^{(t)} - \alpha \nabla J(\theta^{(batch)})
\]

**Pros**:
- Computational efficiency: balances stability and speed.
- Allows parallel computation on GPUs.

**Cons**:
- Needs careful tuning of batch size to optimize performance.

---

\section{Comparison of Gradient Descent Methods}

\begin{center}
    \begin{tabular}{|c|c|c|c|}
        \hline
        Method & Update Frequency & Computational Cost & Convergence Stability \\
        \hline
        Batch GD & Full Dataset & High & Stable \\
        \hline
        SGD & Single Sample & Low & Noisy \\
        \hline
        Mini-Batch GD & Small Subset & Medium & Balanced \\
        \hline
    \end{tabular}
\end{center}

---

\section{Python Implementation}

Below is the Python implementation using `minted` with a gray background.

\begin{minted}[bgcolor=gray!20]{python}
import numpy as np
import matplotlib.pyplot as plt

# Sample data
X = np.array([1, 2, 3, 4, 5])
Y = np.array([2, 4, 5, 4, 5])

# Initialize parameters
beta_0, beta_1 = 0, 0
alpha = 0.01  # Learning rate
epochs = 1000
batch_size = 2  # Mini-batch size

# Store history for visualization
history_beta_0, history_beta_1 = [], []

# Mini-Batch Gradient Descent
for epoch in range(epochs):
    indices = np.random.choice(len(X), batch_size, replace=False)
    X_batch, Y_batch = X[indices], Y[indices]
    
    Y_pred = beta_0 + beta_1 * X_batch
    error = Y_pred - Y_batch

    # Compute gradients
    grad_beta_0 = np.mean(error)
    grad_beta_1 = np.mean(error * X_batch)
    
    # Update parameters
    beta_0 -= alpha * grad_beta_0
    beta_1 -= alpha * grad_beta_1

    history_beta_0.append(beta_0)
    history_beta_1.append(beta_1)

# Print final parameters
print(f"Optimized Parameters: beta_0 = {beta_0:.2f}, beta_1 = {beta_1:.2f}")

# Plot Gradient Descent Progress
plt.figure(figsize=(8, 5))
plt.plot(history_beta_1, label="beta_1", color="red")
plt.plot(history_beta_0, label="beta_0", color="blue")
plt.xlabel("Epochs")
plt.ylabel("Parameter Values")
plt.title("Gradient Descent Optimization Progress")
plt.legend()
plt.show()
\end{minted}

---

\section{Introduction}
This document presents a Python implementation of logistic regression using advanced optimization techniques (Adam, RMSProp, Momentum) and regularization (L1/L2 penalties). 

\section{Optimization Algorithms}

\subsection{Momentum}
Momentum helps accelerate gradient descent in directions with consistent gradients. The update rule is:
\begin{equation}
v_t = \beta v_{t-1} + \eta \nabla J(\theta_t)
\end{equation}
\begin{equation}
\theta_t = \theta_{t-1} - v_t
\end{equation}
where \( v_t \) is velocity, \( \beta \) is the momentum coefficient, and \( \eta \) is the learning rate.

\begin{minted}[bgcolor=gray!20]{python}
v = np.zeros_like(weights)
for epoch in range(epochs):
    gradient = compute_gradient(X, y)
    v = beta1 * v + lr * gradient
    weights -= v
\end{minted}

\subsection{RMSProp}
RMSProp adapts learning rates by maintaining an exponentially decaying average of squared gradients.
\begin{equation}
s_t = \beta s_{t-1} + (1 - \beta) (\nabla J(\theta_t))^2
\end{equation}
\begin{equation}
\theta_t = \theta_{t-1} - \frac{\eta}{\sqrt{s_t} + \epsilon} \nabla J(\theta_t)
\end{equation}

\begin{minted}[bgcolor=gray!20]{python}
s = np.zeros_like(weights)
epsilon = 1e-8
for epoch in range(epochs):
    gradient = compute_gradient(X, y)
    s = beta2 * s + (1 - beta2) * (gradient ** 2)
    weights -= (lr / (np.sqrt(s) + epsilon)) * gradient
\end{minted}

\subsection{Adam Optimization}
Adam combines momentum and RMSProp for more efficient training.
\begin{equation}
m_t = \beta_1 m_{t-1} + (1 - \beta_1) \nabla J(\theta_t)
\end{equation}
\begin{equation}
v_t = \beta_2 v_{t-1} + (1 - \beta_2) (\nabla J(\theta_t))^2
\end{equation}
\begin{equation}
\theta_t = \theta_{t-1} - \frac{\eta}{\sqrt{v_t} + \epsilon} m_t
\end{equation}

\begin{minted}[bgcolor=gray!20]{python}
m, v = np.zeros_like(weights), np.zeros_like(weights)
beta1, beta2 = 0.9, 0.999
for epoch in range(epochs):
    gradient = compute_gradient(X, y)
    m = beta1 * m + (1 - beta1) * gradient
    v = beta2 * v + (1 - beta2) * (gradient ** 2)
    weights -= (lr / (np.sqrt(v) + epsilon)) * m
\end{minted}

\section{Regularization Techniques}

\subsection{L1 Regularization (Lasso)}
L1 regularization encourages sparsity in weights:
\begin{equation}
L1 = \lambda \sum |w_i|
\end{equation}

\begin{minted}[bgcolor=gray!20]{python}
loss += reg_strength * np.sum(np.abs(weights))
gradient += reg_strength * np.sign(weights)
\end{minted}

\subsection{L2 Regularization (Ridge)}
L2 regularization penalizes large weights:
\begin{equation}
L2 = \lambda \sum w_i^2
\end{equation}

\begin{minted}[bgcolor=gray!20, fontsize=\small]{python}
loss += reg_strength * np.sum(weights ** 2)
gradient += 2 * reg_strength * weights
\end{minted}

\section{Logistic Regression with Optimization}
Finally, we integrate these concepts into a logistic regression model.

\begin{minted}[bgcolor=gray!7, fontsize=\small]{python}
import numpy as np

class LogisticRegression:
    def __init__(self, lr=0.01, epochs=1000, optimizer="adam", 
    reg_type=None, reg_strength=0.01):
        self.lr = lr
        self.epochs = epochs
        self.optimizer = optimizer
        self.reg_type = reg_type
        self.reg_strength = reg_strength
        self.beta1, self.beta2 = 0.9, 0.999
        self.epsilon = 1e-8

    def sigmoid(self, z):
        return 1 / (1 + np.exp(-z))

    def compute_loss(self, y_true, y_pred):
        loss = -np.mean(y_true * np.log(y_pred) + (1 - y_true) * np.log(1 - y_pred))
        if self.reg_type == "l1":
            loss += self.reg_strength * np.sum(np.abs(self.weights))
        elif self.reg_type == "l2":
            loss += self.reg_strength * np.sum(self.weights ** 2)
        return loss

    def fit(self, X, y):
        m, n = X.shape
        self.weights = np.zeros(n)
        self.bias = 0
        v, s = np.zeros(n), np.zeros(n)

        for epoch in range(self.epochs):
            linear_model = np.dot(X, self.weights) + self.bias
            y_pred = self.sigmoid(linear_model)

            dw = (1/m) * np.dot(X.T, (y_pred - y))
            db = (1/m) * np.sum(y_pred - y)

            if self.optimizer == "momentum":
                v = self.beta1 * v + self.lr * dw
                self.weights -= v
            elif self.optimizer == "rmsprop":
                s = self.beta2 * s + (dw ** 2)
                self.weights -= (self.lr / (np.sqrt(s) + self.epsilon)) * dw
            elif self.optimizer == "adam":
                v = self.beta1 * v + (1 - self.beta1) * dw
                s = self.beta2 * s + (1 - self.beta2) * (dw ** 2)
                v_corr = v / (1 - self.beta1 ** (epoch + 1))
                s_corr = s / (1 - self.beta2 ** (epoch + 1))
                self.weights -= (self.lr / (np.sqrt(s_corr) + self.epsilon)) * v_corr

            self.bias -= self.lr * db

    def predict(self, X):
        return self.sigmoid(np.dot(X, self.weights) + self.bias)
\end{minted}

\end{document}